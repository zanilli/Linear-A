\documentclass[12pt,a4paper]{article}
\usepackage[utf8]{inputenc}
\usepackage[german]{babel}
\usepackage[T1]{fontenc}
\usepackage{amsmath}
\usepackage{amsfonts}
\usepackage{amssymb}
\usepackage{amsthm}
\usepackage{mathrsfs}
\usepackage{makeidx}
\usepackage{graphicx}
\usepackage{lmodern}
\usepackage{fourier}
\usepackage{enumitem}
\usepackage{titling}



\newtheorem{theorem}{Satz}
\newtheorem{lemma}[theorem]{Lemma}
\newtheorem{proposition}[theorem]{Proposition}
\newtheorem{corollary}[theorem]{Korollar}
\newtheorem{claim}{Behauptung}
\newtheorem*{claim*}{Behauptung}
\newtheorem*{attention}{Achtung!}

\theoremstyle{definition}
\newtheorem{definition}[theorem]{Definition}
\newtheorem{exercise}[theorem]{Übung}

\theoremstyle{remark}
\newtheorem*{remark}{Bemerkung}

\newlist{proofenum}{enumerate}{1}
\setlist[proofenum]{label=(\roman*)}

\renewcommand{\bar}[1]{\overline{#1}}
\renewcommand{\hat}[1]{\widehat{#1}}

\setcounter{section}{6}



\usepackage[left=2cm,right=2cm,top=2cm,bottom=2cm]{geometry}
\author{Felix Zillinger}
\title{Lineare Algebra 2 Skript}
\date{2. Semester}
\begin{document}
	\numberwithin{equation}{subsection}
	\numberwithin{theorem}{subsection}
	\maketitle 
	\tableofcontents

	\newpage	
	
	\section*{Einleitung}
	Dieses Skript ist eine Weiterführung des Skripts von Lineare Algebra I. Das erste Kapitel ist eine Weiterführung des Kapitels über euklidische und unitäre Vektorräume.
	\section{Euklidische und Unitäre Vektorräume}
	\setcounter{subsection}{6}
	\subsection{Spezielle Endomorphismen}
	
	Dieser Abschnitt hat einen vorbereitenden Character. Wir untersuchen hier eine Reihe von Klassen von Endomorphismen mit zusätzlichen Eigenschaften. \\
	Ein Ziel wird es sein, den Spektralsatz für selbstadjungierte Endomorphismen auf seine essentiellen Vorraussetzungen zu untersuchen und zu verallgemeinern. \\
	Im gesamten Abschnitt seien $V, W$ euklidische oder unitäre Vektorräume über $\mathbb{K}(= \mathbb{R}; \mathbb{C})$. \\
	
	\begin{definition}
		Sei $T \in \mathcal{L}(V) := End_{\mathbb{K}}(V)$. $T$ heißt \\
		\begin{itemize}
			\item \textbf{normal}, falls $T^{*} \cdot T = T \cdot T^*$
			\item \textbf{unitär}, falls $T^{*} \cdot T = T \cdot T^* = Id_V$
			\item \textbf{selbstadjungiert}, falls $T^* = T$
			\item \textbf{(Orthogonal)Projektion},  falls $T = T^* = T^2$
		\end{itemize}
	\end{definition}
	
	\begin{remark}
		\begin{itemize}
			\item In Punkt 1 und 2 ist die Existenz von $T^*$ Teil der Vorraussetzung.
			\item In Punkt 2-4 sind die Endomorphismen bereits normal.
			\item Ist $dim(V)<\infty$, so folgt aus $T^* \cdot T = Id_V$ (bzw. $T \cdot T^* = Id_V$) bereits, dass $T$ unitär ist. 
		\end{itemize}
	\end{remark}
	\begin{lemma}
		$T$ normal $\Leftrightarrow$ $T^*$ existiert und $\forall_{x \in V} \  ||Tx|| = ||T^*x || $. \\
	\end{lemma}
	\begin{remark}
		Es folgt dann schon $\forall_{x,y \in V} \ \langle Tx, Ty \rangle = \langle T^*x,T^*y \rangle$
	\end{remark}
	
	\begin{proof}
		$\Rightarrow$: Ist $T$ normal, so folgt für $x \in V$:
		\begin{equation}
			\begin{split}
				||Tx ||^2 &= \langle Tx,Tx \rangle = \langle T^*Tx,x \rangle \\
				&= \langle TT^*x,x\rangle = \langle T^*x,T^*x \rangle \\
				&= ||T^*x ||^2
			\end{split}
		\end{equation}
		$\Leftarrow$: Umgekehrt gelte Gleichung (7.1.1) für $x \in V$. Dann liefert die Polarisierungsidentität (Satz 1.6 in LA I) (exemplarisch für $\mathbb{K} = \mathbb{C}$):
		\begin{equation}
			\begin{split}
				\langle Tx, Ty \rangle &= \frac{1}{4} \sum\limits_{k=0}^3 i^k \langle T(y+i^kx), T(y+i^kx) \rangle \\
				& = \frac{1}{4} \sum\limits_{k=0}^3 i^k \langle T^*(y+i^kx),T^*(y+i^kx) \rangle \\
				&= \langle T^*x,T^*y \rangle
			\end{split}
		\end{equation}
		und somit
		\begin{equation}
			\langle T^*Tx,y \rangle = \langle Tx,Ty \rangle = \langle T^*x, T^*y \rangle = \langle TT^*x,y \rangle
		\end{equation}
		Aus der positiven Definitheit des Skalarprodukts folgt das Lemma.
	\end{proof} 
	
	Wegen ihrer Wichtigkeit stellen wir die folgende Aussage nochmal heraus:
	
	\begin{lemma}
		Für $x,y \in V$ gilt:
		\begin{equation}
			x = y \Leftrightarrow \forall_{z \in V} \langle x,z \rangle = \langle y,z \rangle
		\end{equation}
	\end{lemma}
	
	\begin{proof}
		$\Rightarrow$: klar. \\
		$\Leftarrow$: Es folgt 
		\begin{equation}
			\forall_{z \in V} \langle x-y,z \rangle = 0
		\end{equation}
		Insbesondere mit $z = x-y$ auch $||x-y ||^2 = 0$, also $x=y$.
	\end{proof}
	
	\begin{exercise}
		Sei $T \in \mathcal{L}(V)$. Falls $T = T^*$ und $\forall_{x \in V} \langle Tx,x \rangle = 0$, so ist $T=0$.  \\
		Im Fall $\mathbb{K} = \mathbb{C}$ kann auf die Vorraussetzung $T = T^*$ verzichtet werden, im Falle $\mathbb{K}= \mathbb{R}$ jedoch nicht.
	\end{exercise}
	
	\begin{theorem}
		Sei $T \in \mathcal{L}(V)$ normal. Dann gilt: \\
		\begin{proofenum}
			\item $kerT = kerT^*$
			\item Mit $T$ ist auch $T-\lambda \cdot Id_V$ normal für $\lambda \in \mathbb{K}$. Es gilt $ker(T-\lambda \cdot Id_V) = ker(T^* - \lambda \cdot Id)$. \\ Insbesondere haben $T$ und $T^*$ dieselben Eigenwerte mit gleichen Vielfachheiten.
			\item Sind $\lambda \neq \mu$ verschiedene Eigenwerte von $T$, so stehen die entsprechenden Eigenräume senkrecht: $ker(T-\lambda \cdot Id_V) \perp ker(T-\mu \cdot Id_V)$.
		\end{proofenum}
	\end{theorem}
	
	\begin{proof} 
		\begin{proofenum}
			\item 1 folgt unmittelbar aus Lemma 7.1.2 :\\
				$x \in kerT \Rightarrow Tx = 0 \Rightarrow 0 = ||Tx || = ||T^*x||$, also $x \in kerT^*$
			\item 
				\begin{equation}
					\begin{split}
						(T-\lambda)^*(T - \lambda) &= T^*T- \bar{\lambda}Id_V-\lambda Id_V + |\lambda|^2Id_V \\
						&= TT^*- \bar{\lambda}Id_V-\lambda Id_V + |\lambda|^2Id_V \\
						&= (T-\lambda)(T^*-\lambda)
					\end{split}
				\end{equation}
				Die restlichen Behauptungen folgen nun aus (i).
			\item Sei $Tx = \lambda x$, $Ty = \mu y$. Mit (ii) folgt: 
				\begin{equation}
					\begin{split}
						\lambda \langle x,y \rangle &= \langle \bar{\lambda}x,y \rangle = \langle T^*x,y \rangle \\
						&= \langle x, Ty \rangle = \langle x, \mu y \rangle \\
						&= \langle x,y \rangle \cdot \mu
					\end{split}
				\end{equation}
				also 
				\begin{equation}
					(\lambda - \mu) \langle x,y \rangle = 0 
					\Rightarrow \langle x,y \rangle = o
				\end{equation}
		\end{proofenum}
	\end{proof}
	
	\begin{theorem}
		Für $U \in \mathcal{L}(V)$ sind äquivalent : \\
		\begin{proofenum}
			\item $U$ ist unitär. \\		
			\item $\forall_{x,y \in V} \langle Ux, Uy \rangle = \langle x,y \rangle$ \\
			\item $\forall_{x \in V} || Ux|| = ||x||$
			\item Ist $\{ e_1,...,e_n \}$ ein Orthogonalsystem, so ist auch $\{ Ue_1,...,Ue_n \}$ ein Orthogonalsystem.
		\end{proofenum}
		\begin{remark}
			Ist $U$ unitär, so ist $U$ invertierbar mit $U^{-1} = U^*$.
		\end{remark}
	\end{theorem}
	
	\begin{proof}
		(i) $\Rightarrow$ (ii): \\
		\begin{equation}
			\langle Ux, Uy \rangle = \langle U^*Ux,y \rangle = \langle x,y \rangle
		\end{equation}
		(ii) $\Rightarrow$ (iv): \\
		\begin{equation}
			\langle Ue_i,Ue_j \rangle \overset{(ii)}{=} \langle e_i,e_j \rangle = \delta_{ij}
		\end{equation}		
		(iv) $\Rightarrow$ (iii): \\
		Klar.\\
		(iii) $\Rightarrow$ (ii): \\
		Folgt mit der Polarisierungsformal wie in Beweis von Lemma 7.1.2:
		\begin{equation}
			\begin{split}
				\langle Ux, Uy \rangle &\overset{(\mathbb{K} = \mathbb{C})}{=} \frac{1}{4} \sum\limits_{k=0}^3 i^k \langle U(x+i^ky), U(x+i^ky) \rangle \\
				&= \frac{1}{4} \sum\limits_{k=0}^3 i^k \langle x + i^ky,x+i^ky \rangle \\
				&= \langle x,y \rangle
			\end{split}
		\end{equation}
		(ii) $\Rightarrow$ (i): \\
		\begin{equation}
			\langle U^* Ux,y \rangle = \langle Ux, Uy \rangle = \langle x,y \rangle
		\end{equation}
		$\Rightarrow U^* U = Id_V$ (nach Lemma 7.1.3), $UU^*$ analog
	\end{proof}
	
	\begin{corollary}
		Die Menge der unitären Endomorphismen eines Vektorraums mit Skalarprodukt bilden bzgl. der Komposition eine Gruppe.
	\end{corollary}
	
	\begin{proof}
		Man muss nur noch bemerken, dass für adjungierbare $T_1, T_2 \in \mathcal{L}(V)$ gilt: \\
		$T_1 \circ T_2$ ist adjungierbar und $(T_1 \circ T_2)^* = T_2^* \circ T_1^*$. \\
		Sind $T_1, T_2$ unitär, so folgt :
		\begin{equation}
			(T_1 \cdot T_2)^* T_1 T_2 = T_2^*(T_1^*T_1)T_2 = T_2^*T_2=Id_V
		\end{equation}
	\end{proof}	
	
	Unitarität lässt sich in Koordinaten sehr leicht testen:
	
	\begin{theorem}
		Sei $V$ ein endlich-dimensionaler Vektorraum mit Skalarprodukt. \\
		Dann ist $T \in \mathcal{L}(V)$ genau dann normal (unitär), wenn bzgl. einer Orthonormalbasis $B=\{e_1,...,e_n  \}$ die Matrix $M_B(T)$ diese Eigenschaft hat.
	\end{theorem}
	
	\begin{proof}
		Der Satz folgt unmittelbar aus Satz 4.2. Für Orthonormalbasen gilt: \\
		\begin{equation}
			M_B(T)^* = M_B(T^*)
		\end{equation}
		Wobei $M_B(T)*$ die Matrixadjunktion bezeichnet.
	\end{proof}		
	
	\newpage	
	
 	\subsection{Der Spektralsatz für normale Endomorphismen}
	\begin{remark}
		Der komplexe Fall ist hier wesentlich einfacher als der reelle Fall.
	\end{remark}
	
	\begin{theorem}{(Spektralsatz für normale Endomorphismen, $\mathbb{K} = \mathbb{C}$)}
		Sei $V$ ein unitärer Vektorraum, $\mathbb{K} = \mathbb{C}$, $dim V = n < \infty$. Dann ist $T \in \mathcal{L}(V)$ genau dann normal, wenn es zu $T$ eine Orthonormalbasis von $V$ gibt, welche nur aus Eigenvektoren von $T$ besteht.
	\end{theorem}
	
	\begin{proof}
		(Analog zum beweis von Satz 5.3) \\
		Wir beginnen mit der einfacher Implikation: \\
		Sei $B = \{ e_1,...,e_n \}$ eine Orthonormalbasis von $V$ mit $Te_j = \lambda_j e_j$. Das heißt: \\
		\begin{equation}
			M_B(T) = 
			\begin{pmatrix}
				\lambda_1 & \dots & 0 \\
				\vdots & \ddots & \vdots \\
				0 & \dots & \lambda_n
			\end{pmatrix}
		\end{equation}
		Nach Satz 4.2 ist 
		\begin{equation}
			M_B(T^*) = M_B(T)^* = 
			\begin{pmatrix}
				\bar{\lambda}_1 & \dots & 0 \\
				\vdots & \ddots & \vdots \\
				0 & \dots & \bar{\lambda}_n
			\end{pmatrix}
		\end{equation}
		das heißt also $T^*e_j = \bar{\lambda}_j e_j$ und folglich
		\begin{equation}
			M_B(T^*T) = 
			\begin{pmatrix}
				|\lambda_1|^2 & \dots & 0 \\
				\vdots & \ddots & \vdots \\
				0 & \dots & |\lambda_n|^2
			\end{pmatrix}
			= M_B(TT^*)
		\end{equation}
		folglich ist $TT^* = T^* T$, d.h. $T$ ist normal. \\
		Umgekehrt sei $T$ normal: \\
		 Da $\mathbb{K} = \mathbb{C}$, existiert zu $T$ ein normierter Eigenvektor $e_1$ mit $||e_1 ||=1, Te_1 = \lambda_1 e_1$ und da $T$ normal ist, gilt $T^* e_1 = \bar{\lambda}_1 e_1 $. \\
		 Wir führen nun eine Induktion nach $n = dimV$ und nehmen an, die Behauptung gelte für alle Dimensionen $<n$: \\
		 $U = \langle e_1 \rangle^{\perp} \subset V$ ist ein unitärer Vektorraum der Dimension $n-1$. \\
		 Wir zeigen:
		 \begin{claim}
		 	$T|_U \in End_{\mathbb{C}}(U)$ ist normal.
		 \end{claim}
		 Die Induktionsvorraussetzung liefert die Existenz einer Orthonormalbasis $\{ e_2,...,e_n \}$ von $U$ mit $Te_j = \lambda_j e_j$. $\{ e_1,e_2,...,e_n \}$ ist dann die gewünschte Basis. \\
		 Es bleibt Behauptung 1 zu zeigen: \\
		 Sei $x \in U$. Dann ist 
		 \begin{equation}
		 	\langle Tx,e_1 \rangle = \langle x, T^* e_1 \rangle = \bar{\lambda}_1 \langle x, e_1 \rangle = 0
		 \end{equation}
		 also $T(U) \subset U$. Analog sieht man $T^*(U) \subset U$. \\
		 Dann folgt aber $(T|_U)^* = T^*|_U$ und daher
		 \begin{equation}
		 	(T|_U)^*T|_U = T^*T|_U = TT^*|_U= T|_U(T|_U)^*
		 \end{equation}
	\end{proof}
	
	\newpage	
	
	\textbf{Die Komplexifizierung eines rellen Vektorraumes} \\
	Der reelle Fall bedarf einiger Vorbereitung. Zur Motivation betrachten wir den $V_{\mathbb{C}} = \mathbb{C}^n$ mit dem Skalarprodukt 
	\begin{equation}
		\langle z,w \rangle = \sum\limits_{j=1}^n \bar{z}_j w_j
	\end{equation}
	$V_{\mathbb{C}}$ ist gleichzeitig ein reeller Vektorraum, indem man die Zerlegung in Real- und Imaginärtiel komponentenweise vornimmt: \\
	\begin{proofenum}
		\item Jedes $z \in \mathbb{C}^n$ besitzt eine eindeutige Zerlegung $z = x+iy$ mit $x,y \in \mathbb{R}^n$. Offenbar ist $V = \mathbb{R}^n$ ein $\mathbb{R}$-Vektorraum.
		\item Es gibt eine $\mathbb{C}$antilineare Involution \ $\bar{\cdot}: V_{\mathbb{C}} \rightarrow V_{\mathbb{C}}$ mit $V = \{ z \in V_{\mathbb{C}} | \bar{z} = z \}$ (klar: \ $\bar{\cdot}$ \ ist die komplexe Konjugation ).
		\item Ist $T \in End_{\mathbb{R}}$, so besitzt $T$ eine eindeutige Fortsetzung $\hat{T} \in End_{\mathbb{C}}(V_{\mathbb{C}})$ (klar: dies ist die gleiche Matrix). \\
		Es gilt, da $M_B(T)$ reell ist $\bar{T(z)} = T(\bar{z})$.
		\item Es gibt genau ein hermitesches Skalarprodukt auf $V_{\mathbb{C}}$, welches das (Standard)skalarprodukt fortsetzt (klar: dies ist das Standardskalarprodukt auf $\mathbb{C}^n$, Eindeutigkeit: Übung)
	\begin{attention}
		Für $z = x + iy$, $\tilde{z} = \tilde{x} + i \tilde{y} \in \mathbb{C}^n$ ist
		\begin{equation}
			\begin{split}
				\langle z, \tilde{z} \rangle &= \sum\limits_{j=1}^n \bar{z}_j \tilde{z}_j \\
				&= \sum\limits_{j=1}^n (x_j \tilde{x}_j+y_j \tilde{y}_j)+i(-y_j \tilde{x}_j+x_j \tilde{y}_j) \\
				&= (\langle x,\tilde{x} \rangle_{\mathbb{R}^n} + \langle y, \tilde{y} \rangle_{\mathbb{R}^n})+i(-\langle y,\tilde{x} \rangle_{\mathbb{R}^n} + \langle x,\tilde{y} \rangle_{\mathbb{R}^n})
			\end{split}
		\end{equation}
	\end{attention}
		\item Ist $T \in End_{\mathbb{R}}(\mathbb{R}^n)$ normal, so ist auch $\hat{T} \in End_{\mathbb{C}}(\mathbb{C}^n)$ normal.
	\end{proofenum}
	Die hier gemachten Bemerkungen sind nicht auf den $\mathbb{C}^n$ / $\mathbb{R}^n$ begrenzt. Allgemein gilt: \\
	\begin{theorem}
		Sei $V$ ein euklidischer Vektorraum (Vektorraum mit Skalarprodukt über $\mathbb{R}$). Darum existiert ein komplexer Vektorraum $V_{\mathbb{C}}$, so dass (i)-(v) gelten. (Details: Übung)
	\end{theorem}
	Nach dieser Vorbereitung können wir den Spektralsatz für normale Endomorphismen über $\mathbb{K} = \mathbb{R}$ formulieren. 
	\newpage
	\begin{theorem}{Spektralsatz für normale Endomorphismen,  $\mathbb{K} = \mathbb{R}$}
		Sei $V$ ein euklidischer Vektorraum. Dann ist $T \in \mathcal{L}(V)$ genau dann normal, wenn es für $T$ eine Orthonormalbasis $B$ von $V$ gibt, bzgl. der $M_B(T)$ die Gestalt \\
		\begin{equation}
			M_B(T) =
			\begin{pmatrix}
				\lambda_1 & \dots & 0 &0 & \dots & 0 \\
				\vdots &\ddots & \vdots & \vdots & \ddots& \vdots\\
				0 & \dots & \lambda_r & 0 &\dots & 0 \\
				0 & \dots & 0 & \mathcal{Q}_1 & \dots & 0 \\
				\vdots & \ddots & \vdots & \vdots & \ddots & \vdots \\
				0 & \dots & 0 & 0 & \dots & \mathcal{Q}_r
			\end{pmatrix}	
		\end{equation}
		Dabei sind $\lambda_1,...,\lambda_r \in \mathbb{R}$ die reellen Eigenwerte von $T$. Jedes $\mathcal{Q}_j \in M(2,\mathbb{R})$ ist von der Gestalt
		\begin{equation}
			\mathcal{Q}_j = 
			\begin{pmatrix}
				\alpha_j & \beta_j \\
				- \beta_j & \alpha_j
			\end{pmatrix}
		\end{equation}
		Dabei sind $\mu_j = \alpha_j + i\beta_j $ genau die komplexen Eigenwerte von $\hat{T}$ (=komplexe Nullstellen des charakteristischen Polynoms von $T$).
	\end{theorem}
	\begin{proof}
		Wir orientieren uns am Beweis von Satz 7.8.1 mit entsprechenden Modifikationen. \\
		Zunächst nehmen wir an, dass $M_B(T)$ obige Gestalt hat. Dann ist wiederum nach Satz 7.4.2
		\begin{equation}
			M_B(T^*) =
			\begin{pmatrix}
				\bar{\lambda}_1 & \dots & 0 &0 & \dots & 0 \\
				\vdots &\ddots & \vdots & \vdots & \ddots& \vdots\\
				0 & \dots & \bar{\lambda}_r & 0 &\dots & 0 \\
				0 & \dots & 0 & \mathcal{Q}_1^* & \dots & 0 \\
				\vdots & \ddots & \vdots & \vdots & \ddots & \vdots \\
				0 & \dots & 0 & 0 & \dots & \mathcal{Q}_r^*
			\end{pmatrix}
		\end{equation}
		also gilt 
		\begin{equation}
			M_B(T^*T) =
			\begin{pmatrix}
				|\lambda_1|^2 & \dots & 0 &0 & \dots & 0 \\
				\vdots &\ddots & \vdots & \vdots & \ddots& \vdots\\
				0 & \dots & |\lambda_r|^2 & 0 &\dots & 0 \\
				0 & \dots & 0 & \mathcal{Q}_1^* \mathcal{Q}_1 & \dots & 0 \\
				\vdots & \ddots & \vdots & \vdots & \ddots & \vdots \\
				0 & \dots & 0 & 0 & \dots & \mathcal{Q}_r^* \mathcal{Q}_r
			\end{pmatrix}
		\end{equation}
		Wenn man sich eines der Kästchen anschaut sieht man
		\begin{equation}
			\begin{split}
				\mathcal{Q}_j^* \mathcal{Q}_j &= \begin{pmatrix}
					\alpha_j & -\beta_j \\
					\beta_j & \alpha_j
				\end{pmatrix} \cdot \begin{pmatrix}
					\alpha_j & \beta_j \\
					-\beta_j & \alpha_j
				\end{pmatrix} = \begin{pmatrix}
					\alpha_j^2 + \beta_j^2 & 0 \\
					0 & \alpha_j^2 + \beta_j^2
				\end{pmatrix} \\
				&= \mathcal{Q}_j \mathcal{Q}_j^* 
			\end{split}
		\end{equation}
		Folglich gilt 
		\begin{equation}
			M_B(T^*T) = M_B(TT^*)
		\end{equation}
		also ist $T$ normal. \\
		Wir berechnen das charakteristische Polynom zunächst für $\mathcal{Q}_j$: \\
		\begin{equation}
			\begin{split}
				det\begin{pmatrix}
					X - \alpha_j & \beta_j \\
					\beta_j & X- \alpha_j
				\end{pmatrix} &= X^2-2\alpha_jX+\alpha_j^2+\beta_j^2 \\
				&= X^2 - 2 Re(\mu_j)X+ |\mu_j|^2, \ \ \  \mu_j := \alpha_j + i \beta_j \\
				&= (X- \mu_j) (X- \bar{\mu}_j)
			\end{split}
		\end{equation}
		Das heißt das charakteristische Polynom von T ist gegeben durch
		\begin{equation}
			\chi_T(X) = \prod\limits_{j=1}^r (X-\lambda_j) \prod\limits_{j=1}^s (X-\mu_j)(X-\bar{\mu}_j)
		\end{equation}
		Dies beendet den Beweis der Richtung $\Leftarrow$. \\
		Zum Beweis der Umkehrung wenden wir Induktion über $n = dim(V)$. Für $n=1$ ist jeder Endomorphismus normal und schon in Diagonalgestalt, es ist also nichts zu zeigen. \\
		Die Behauptung sei bewiesen für Dimensionen $<n$ und sei nun $n=dimV$ und $T \in End_{\mathbb{R}}(V)$ normal gegeben. Dann gibt es die folgenden zwei Fälle: \\
		\begin{itemize}
			\item \textbf{1. Fall:} $T$ besitzt einen reellen Eigenwert.\\
				Dann erfolgt der Beweis wie in Satz 7.8.1. Man nehme einen Eigenvektor $e_1$, $||e_1||=1, \ Te_1 = \lambda_1 e_1, \ \lambda_1 \in \mathbb{R}$ und wendet die Induktionsvoraussetzung auf den normalen Endomorphismus $T|_{\langle e_1 \rangle^{\perp}}$ an.
			\item \textbf{2. Fall:} $T$ besitzt keinen reellen Eigenwert. \\
			Nun betrachten wir $\hat{T} \in End_{\mathbb{C}}(V_{\mathbb{C}})$ (Satz 7.8.2).
			$\hat{T}$ ist normal mit gleichem charakteristischem Polynom wie $T$ (Übung). Sei also $\mu = \alpha + i \beta, \ \beta \neq 0$ ein komplexer Eigenwert von $\hat{T}$ mit normiertem Eigenvektor $\xi \in V_{\mathbb{C}}$. Es ist nach Satz 7.8.2 
			\begin{equation}
				\hat{T}(\overline{\xi})=\overline{T(\xi)}= \overline{\mu \cdot \xi} = \overline{\mu} \cdot \overline{\xi}
			\end{equation}
			und da $\mu \neq \bar{\mu}$ und $\hat{T}$ normal ist, folgt aus Satz 7.7.4, dass $\xi \perp \bar{\xi}$, also mit 
			\begin{equation}
				\begin{split}
					\xi &= \zeta + i \eta \\
					\zeta :&= \frac{1}{\sqrt{2}}(\xi + \bar{\xi}) \\
					\eta :&= \frac{1}{\sqrt{2}i}(\xi - \bar{\xi}) \in V \\
					||\zeta||^2&= \frac{1}{2} (||\xi||^2+||\bar{\xi}||^2) =1 \\
					||\eta ||^2 &= 1 \ analog \\
					\langle \zeta, \eta \rangle &= \frac{1}{2i}\langle \xi + \bar{\xi}, \xi - \bar{\xi} \rangle \\
					&= \frac{1}{2i}(||\xi||^2-||\bar{\xi}||^2) \\
					&=0
				\end{split}
			\end{equation}
			Übung: $||\xi||=||\bar{\xi}||$. \\
			Das heißt, die Vektoren $\zeta,\eta \in V$ sind orthonormal und spannen einen 2-dimensionalen Unterraum von V auf. \\
			Bezüglich der Basis $\langle \zeta, \eta \rangle$ dieses Unterraums hat $T$ die Matrix
			\begin{equation}
				\begin{split}
					T\zeta &= \hat{T}(\frac{1}{\sqrt{2}}(\xi + \bar{\xi})) \\
					&= \frac{1}{\sqrt{2}}((\alpha+i\beta)\xi+(\alpha-i\beta)\bar{\xi}) \\
					&= \alpha\zeta - \beta \eta \\
					T\eta &= \hat{T}(\frac{1}{\sqrt{2}i}(\xi-\bar{\xi})) \\
					&= \frac{1}{\sqrt{2}i}((\alpha+i\beta)\xi-(\alpha-i\beta)\bar{\xi}) \\
					&= \alpha\eta + \beta\zeta \\
					&also \ gilt \\
					M_{\langle\zeta,\eta \rangle}(T) &= 
					\begin{pmatrix}
						\alpha & \beta \\
						-\beta & \alpha
					\end{pmatrix}
				\end{split}
			\end{equation}
			Da $T$ normal ist, sieht man nun wie im Beweis von Satz 7.8.1, dass T auch den Unterraum $U = \langle \zeta,\eta \rangle^{\perp}$ invariant lässt und $T|_U$ ebenfalls normal ist. Nun wendet man die Induktionsvoraussetzung auf $U$ an und ist fertig
		\end{itemize}
	\end{proof}
\end{document}