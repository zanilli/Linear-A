\documentclass[12pt,a4paper]{article}
\usepackage[utf8]{inputenc}
\usepackage[german]{babel}
\usepackage[T1]{fontenc}
\usepackage{amsmath}
\usepackage{amssymb}
\usepackage{amsthm}
\usepackage{makeidx}
\usepackage{enumitem}
\usepackage{titling}
\usepackage{tikz-cd}
\usepackage{eqparbox}
\usetikzlibrary{quotes,babel,angles}

\newtheorem{theorem}{Satz}
\newtheorem{lemma}[theorem]{Lemma}
\newtheorem{proposition}[theorem]{Proposition}
\newtheorem{corollary}[theorem]{Korollar}
\newtheorem{claim}{Behauptung}
\newtheorem{axiom}{Axiom}
\newtheorem*{claim*}{Behauptung}
\newtheorem*{attention}{Achtung!}
\newtheorem*{example}{Beispiel}

\theoremstyle{definition}
\newtheorem{definition}[theorem]{Definition}
\newtheorem*{exercise}{Übung}

\theoremstyle{remark}
\newtheorem*{remark}{Bemerkung}

\newlist{proofenum}{enumerate}{1}
\setlist[proofenum]{label=(\roman*)}

\renewcommand{\bar}[1]{\overline{#1}}
\renewcommand{\hat}[1]{\widehat{#1}}

\DeclareMathOperator{\Hom}{Hom}
\DeclareMathOperator{\Coker}{Coker}
\DeclareMathOperator{\Ker}{Ker}
\DeclareMathOperator{\Img}{Im}
\DeclareMathOperator{\Ext}{Ext}
\DeclareMathOperator{\Tor}{Tor}
\DeclareMathOperator{\Ann}{Ann}
\DeclareMathOperator{\End}{End}
\DeclareMathOperator{\rg}{rg}
\DeclareMathOperator{\Abb}{Abb}

\setcounter{section}{6}


\usepackage{cmbright}


\usepackage[left=2cm,right=2cm,top=2cm,bottom=2cm]{geometry}
\author{Felix Zillinger}
\title{Lineare Algebra 2 Skript}
\date{2. Semester}
\begin{document}
	\numberwithin{equation}{subsection}
	\numberwithin{theorem}{subsection}
	\maketitle 
	\tableofcontents

	\newpage	
	
	\section*{Einleitung}
	Dieses Skript ist eine Weiterführung des Skripts von Lineare Algebra I. Das erste Kapitel ist eine Weiterführung des Kapitels über euklidische und unitäre Vektorräume.
	\section{Euklidische und Unitäre Vektorräume}
	\setcounter{subsection}{6}
	\subsection{Spezielle Endomorphismen}
	
	Dieser Abschnitt hat einen vorbereitenden Character. Wir untersuchen hier eine Reihe von Klassen von Endomorphismen mit zusätzlichen Eigenschaften. \\
	Ein Ziel wird es sein, den Spektralsatz für selbstadjungierte Endomorphismen auf seine essentiellen Vorraussetzungen zu untersuchen und zu verallgemeinern. \\
	Im gesamten Abschnitt seien $V, W$ euklidische oder unitäre Vektorräume über $\mathbb{K}(= \mathbb{R}; \mathbb{C})$. \\
	
	\begin{definition}
		Sei $T \in \mathcal{L}(V) := \End_{\mathbb{K}}(V)$. $T$ heißt \\
		\begin{itemize}
			\item \textbf{normal}, falls $T^{*} \cdot T = T \cdot T^*$
			\item \textbf{unitär}, falls $T^{*} \cdot T = T \cdot T^* = Id_V$
			\item \textbf{selbstadjungiert}, falls $T^* = T$
			\item \textbf{(Orthogonal)Projektion},  falls $T = T^* = T^2$
		\end{itemize}
	\end{definition}
	
	\begin{remark}
		\begin{itemize}
			\item In Punkt 1 und 2 ist die Existenz von $T^*$ Teil der Vorraussetzung.
			\item In Punkt 2-4 sind die Endomorphismen bereits normal.
			\item Ist $\dim (V)<\infty$, so folgt aus $T^* \cdot T = Id_V$ (bzw. $T \cdot T^* = Id_V$) bereits, dass $T$ unitär ist. 
		\end{itemize}
	\end{remark}
	\begin{lemma}
		$T$ normal $\Leftrightarrow$ $T^*$ existiert und $\forall_{x \in V} \  ||Tx|| = ||T^*x || $. \\
	\end{lemma}
	\begin{remark}
		Es folgt dann schon $\forall_{x,y \in V} \ \langle Tx, Ty \rangle = \langle T^*x,T^*y \rangle$
	\end{remark}
	
	\begin{proof}
		$\Rightarrow$: Ist $T$ normal, so folgt für $x \in V$:
		\begin{equation}
			\begin{split}
				||Tx ||^2 &= \langle Tx,Tx \rangle = \langle T^*Tx,x \rangle \\
				&= \langle TT^*x,x\rangle = \langle T^*x,T^*x \rangle \\
				&= ||T^*x ||^2
			\end{split}
		\end{equation}
		$\Leftarrow$: Umgekehrt gelte Gleichung (7.1.1) für $x \in V$. Dann liefert die Polarisierungsidentität (Satz 7.1.6 in LA I) (exemplarisch für $\mathbb{K} = \mathbb{C}$):
		\begin{equation}
			\begin{split}
				\langle Tx, Ty \rangle &= \frac{1}{4} \sum\limits_{k=0}^3 i^k \langle T(y+i^kx), T(y+i^kx) \rangle \\
				& = \frac{1}{4} \sum\limits_{k=0}^3 i^k \langle T^*(y+i^kx),T^*(y+i^kx) \rangle \\
				&= \langle T^*x,T^*y \rangle
			\end{split}
		\end{equation}
		und somit
		\begin{equation}
			\langle T^*Tx,y \rangle = \langle Tx,Ty \rangle = \langle T^*x, T^*y \rangle = \langle TT^*x,y \rangle
		\end{equation}
		Aus der positiven Definitheit des Skalarprodukts folgt das Lemma.
	\end{proof} 
	
	Wegen ihrer Wichtigkeit stellen wir die folgende Aussage nochmal heraus:
	
	\begin{lemma}
		Für $x,y \in V$ gilt:
		\begin{equation}
			x = y \Leftrightarrow \forall_{z \in V} \langle x,z \rangle = \langle y,z \rangle
		\end{equation}
	\end{lemma}
	
	\begin{proof}
		$\Rightarrow$: klar. \\
		$\Leftarrow$: Es folgt 
		\begin{equation}
			\forall_{z \in V} \langle x-y,z \rangle = 0
		\end{equation}
		Insbesondere mit $z = x-y$ auch $||x-y ||^2 = 0$, also $x=y$.
	\end{proof}
	
	\begin{exercise}
		Sei $T \in \mathcal{L}(V)$. Falls $T = T^*$ und $\forall_{x \in V} \langle Tx,x \rangle = 0$, so ist $T=0$.  \\
		Im Fall $\mathbb{K} = \mathbb{C}$ kann auf die Vorraussetzung $T = T^*$ verzichtet werden, im Falle $\mathbb{K}= \mathbb{R}$ jedoch nicht.
	\end{exercise}
	
	\begin{theorem}
		Sei $T \in \mathcal{L}(V)$ normal. Dann gilt: \\
		\begin{proofenum}
			\item $\ker T = \ker T^*$
			\item Mit $T$ ist auch $T-\lambda \cdot Id_V$ normal für $\lambda \in \mathbb{K}$. Es gilt $\ker (T-\lambda \cdot Id_V) = \ker (T^* - \lambda \cdot Id)$. \\ Insbesondere haben $T$ und $T^*$ dieselben Eigenwerte mit gleichen Vielfachheiten.
			\item Sind $\lambda \neq \mu$ verschiedene Eigenwerte von $T$, so stehen die entsprechenden Eigenräume senkrecht: $\ker (T-\lambda \cdot Id_V) \perp \ker (T-\mu \cdot Id_V)$.
		\end{proofenum}
	\end{theorem}
	
	\begin{proof} 
		\begin{proofenum}
			\item 1 folgt unmittelbar aus Lemma 7.1.2 :\\
				$x \in \ker T \Rightarrow Tx = 0 \Rightarrow 0 = ||Tx || = ||T^*x||$, also $x \in \ker T^*$
			\item 
				\begin{equation}
					\begin{split}
						(T-\lambda)^*(T - \lambda) &= T^*T- \bar{\lambda}Id_V-\lambda Id_V + |\lambda|^2Id_V \\
						&= TT^*- \bar{\lambda}Id_V-\lambda Id_V + |\lambda|^2Id_V \\
						&= (T-\lambda)(T^*-\lambda)
					\end{split}
				\end{equation}
				Die restlichen Behauptungen folgen nun aus (i).
			\item Sei $Tx = \lambda x$, $Ty = \mu y$. Mit (ii) folgt: 
				\begin{equation}
					\begin{split}
						\lambda \langle x,y \rangle &= \langle \bar{\lambda}x,y \rangle = \langle T^*x,y \rangle \\
						&= \langle x, Ty \rangle = \langle x, \mu y \rangle \\
						&= \langle x,y \rangle \cdot \mu
					\end{split}
				\end{equation}
				also 
				\begin{equation}
					(\lambda - \mu) \langle x,y \rangle = 0 
					\Rightarrow \langle x,y \rangle = o
				\end{equation}
		\end{proofenum}
	\end{proof}
	
	\begin{theorem}
		Für $U \in \mathcal{L}(V)$ sind äquivalent : \\
		\begin{proofenum}	
			\item $\forall_{x,y \in V} \langle Ux, Uy \rangle = \langle x,y \rangle$ \\
			\item $\forall_{x \in V} || Ux|| = ||x||$
			\item Ist $\{ e_1,...,e_n \}$ ein Orthogonalsystem, so ist auch $\{ Ue_1,...,Ue_n \}$ ein Orthogonalsystem.
		\end{proofenum}
		\begin{remark}
			Ist $U$ unitär, so gelten (i)-(iii).Ist $\dim (V)<\infty$, so implizieren (i),(ii) oder (iii) auch, dass $U$ unitär ist.
		\end{remark}
	\end{theorem}
	
	\begin{proof}
		(i) $\Rightarrow$ (iii): \\
		\begin{equation}
			\langle Ue_i,Ue_j \rangle \overset{(i)}{=} \langle e_i,e_j \rangle = \delta_{ij}
		\end{equation}		
		(iii) $\Rightarrow$ (ii): \\
		Klar.\\
		(ii) $\Rightarrow$ (i): \\
		Folgt aus der Polarisierungsformel.
	\end{proof}
	
	\begin{corollary}
		Die Menge der unitären Endomorphismen eines Vektorraums mit Skalarprodukt bilden bzgl. der Komposition eine Gruppe.
	\end{corollary}
	
	\begin{proof}
		Man muss nur noch bemerken, dass für adjungierbare $T_1, T_2 \in \mathcal{L}(V)$ gilt: \\
		$T_1 \circ T_2$ ist adjungierbar und $(T_1 \circ T_2)^* = T_2^* \circ T_1^*$. \\
		Sind $T_1, T_2$ unitär, so folgt :
		\begin{equation}
			(T_1 \cdot T_2)^* T_1 T_2 = T_2^*(T_1^*T_1)T_2 = T_2^*T_2=Id_V
		\end{equation}
		Beweis der Bemerkung: \\
		Ist $U$ unitär, so folgt für $x,y \in V$:
		\begin{equation}
			\langle Ux,Uy \rangle = \langle U^*Ux,y \rangle =\langle x,y \rangle
		\end{equation}
		Ist $\dim (V)<\infty$ und gelte für $U \in \mathcal{L}(V)$ die äquivalenten Aussagen (i),(ii),(iii), so liefert Satz 7.4.2 die Existenz von $U^*$ und somit (i).
	\end{proof}	
	
	Unitarität lässt sich in Koordinaten sehr leicht testen:
	
	\begin{theorem}
		Sei $V$ ein endlich-dimensionaler Vektorraum mit Skalarprodukt. \\
		Dann ist $T \in \mathcal{L}(V)$ genau dann normal (unitär), wenn bzgl. einer Orthonormalbasis $B=\{e_1,...,e_n  \}$ die Matrix $M_B(T)$ diese Eigenschaft hat.
	\end{theorem}
	
	\begin{proof}
		Der Satz folgt unmittelbar aus Satz 4.2. Für Orthonormalbasen gilt: \\
		\begin{equation}
			M_B(T)^* = M_B(T^*)
		\end{equation}
		Wobei $M_B(T)*$ die Matrixadjunktion bezeichnet.
	\end{proof}		
	
	\newpage	
	
 	\subsection{Der Spektralsatz für normale Endomorphismen}
	\begin{remark}
		Der komplexe Fall ist hier wesentlich einfacher als der reelle Fall.
	\end{remark}
	
	\begin{theorem}{(Spektralsatz für normale Endomorphismen, $\mathbb{K} = \mathbb{C}$)}
		Sei $V$ ein unitärer Vektorraum, $\mathbb{K} = \mathbb{C}$, $\dim V = n < \infty$. Dann ist $T \in \mathcal{L}(V)$ genau dann normal, wenn es zu $T$ eine Orthonormalbasis von $V$ gibt, welche nur aus Eigenvektoren von $T$ besteht.
	\end{theorem}
	
	\begin{proof}
		(Analog zum beweis von Satz 5.3) \\
		Wir beginnen mit der einfacher Implikation: \\
		Sei $B = \{ e_1,...,e_n \}$ eine Orthonormalbasis von $V$ mit $Te_j = \lambda_j e_j$. Das heißt: \\
		\begin{equation}
			M_B(T) = 
			\begin{pmatrix}
				\lambda_1 & \dots & 0 \\
				\vdots & \ddots & \vdots \\
				0 & \dots & \lambda_n
			\end{pmatrix}
		\end{equation}
		Nach Satz 4.2 ist 
		\begin{equation}
			M_B(T^*) = M_B(T)^* = 
			\begin{pmatrix}
				\bar{\lambda}_1 & \dots & 0 \\
				\vdots & \ddots & \vdots \\
				0 & \dots & \bar{\lambda}_n
			\end{pmatrix}
		\end{equation}
		das heißt also $T^*e_j = \bar{\lambda}_j e_j$ und folglich
		\begin{equation}
			M_B(T^*T) = 
			\begin{pmatrix}
				|\lambda_1|^2 & \dots & 0 \\
				\vdots & \ddots & \vdots \\
				0 & \dots & |\lambda_n|^2
			\end{pmatrix}
			= M_B(TT^*)
		\end{equation}
		folglich ist $TT^* = T^* T$, d.h. $T$ ist normal. \\
		Umgekehrt sei $T$ normal: \\
		 Da $\mathbb{K} = \mathbb{C}$, existiert zu $T$ ein normierter Eigenvektor $e_1$ mit $||e_1 ||=1, Te_1 = \lambda_1 e_1$ und da $T$ normal ist, gilt $T^* e_1 = \bar{\lambda}_1 e_1 $. \\
		 Wir führen nun eine Induktion nach $n = \dim V$ und nehmen an, die Behauptung gelte für alle Dimensionen $<n$: \\
		 $U = \langle e_1 \rangle^{\perp} \subset V$ ist ein unitärer Vektorraum der Dimension $n-1$. \\
		 Wir zeigen:
		 \begin{claim*}
		 	$T|_U \in \End_{\mathbb{C}}(U)$ ist normal.
		 \end{claim*}
		 Die Induktionsvorraussetzung liefert die Existenz einer Orthonormalbasis $\{ e_2,...,e_n \}$ von $U$ mit $Te_j = \lambda_j e_j$. $\{ e_1,e_2,...,e_n \}$ ist dann die gewünschte Basis. \\
		 Es bleibt Behauptung 1 zu zeigen: \\
		 Sei $x \in U$. Dann ist 
		 \begin{equation}
		 	\langle Tx,e_1 \rangle = \langle x, T^* e_1 \rangle = \bar{\lambda}_1 \langle x, e_1 \rangle = 0
		 \end{equation}
		 also $T(U) \subset U$. Analog sieht man $T^*(U) \subset U$. \\
		 Dann folgt aber $(T|_U)^* = T^*|_U$ und daher
		 \begin{equation}
		 	(T|_U)^*T|_U = T^*T|_U = TT^*|_U= T|_U(T|_U)^*
		 \end{equation}
	\end{proof}
	
	\newpage	
	
	\textbf{Die Komplexifizierung eines rellen Vektorraumes} \\
	Der reelle Fall bedarf einiger Vorbereitung. Zur Motivation betrachten wir den $V_{\mathbb{C}} = \mathbb{C}^n$ mit dem Skalarprodukt 
	\begin{equation}
		\langle z,w \rangle = \sum\limits_{j=1}^n \bar{z}_j w_j
	\end{equation}
	$V_{\mathbb{C}}$ ist gleichzeitig ein reeller Vektorraum, indem man die Zerlegung in Real- und Imaginärteil komponentenweise vornimmt: \\
	\begin{proofenum}
		\item Jedes $z \in \mathbb{C}^n$ besitzt eine eindeutige Zerlegung $z = x+iy$ mit $x,y \in \mathbb{R}^n$. Offenbar ist $V = \mathbb{R}^n$ ein $\mathbb{R}$-Vektorraum.
		\item Es gibt eine $\mathbb{C}$antilineare Involution \ $\bar{\cdot}: V_{\mathbb{C}} \rightarrow V_{\mathbb{C}}$ mit $V = \{ z \in V_{\mathbb{C}} | \bar{z} = z \}$ (klar: \ $\bar{\cdot}$ \ ist die komplexe Konjugation ).
		\item Ist $T \in \End_{\mathbb{R}}$, so besitzt $T$ eine eindeutige Fortsetzung $\hat{T} \in \End_{\mathbb{C}}(V_{\mathbb{C}})$ (klar: dies ist die gleiche Matrix). \\
		Es gilt, da $M_B(T)$ reell ist $\bar{T(z)} = T(\bar{z})$.
		\item Es gibt genau ein hermitesches Skalarprodukt auf $V_{\mathbb{C}}$, welches das (Standard)skalarprodukt fortsetzt (klar: dies ist das Standardskalarprodukt auf $\mathbb{C}^n$, Eindeutigkeit: Übung)
	\begin{attention}
		Für $z = x + iy$, $\tilde{z} = \tilde{x} + i \tilde{y} \in \mathbb{C}^n$ ist
		\begin{equation}
			\begin{split}
				\langle z, \tilde{z} \rangle &= \sum\limits_{j=1}^n \bar{z}_j \tilde{z}_j \\
				&= \sum\limits_{j=1}^n (x_j \tilde{x}_j+y_j \tilde{y}_j)+i(-y_j \tilde{x}_j+x_j \tilde{y}_j) \\
				&= (\langle x,\tilde{x} \rangle_{\mathbb{R}^n} + \langle y, \tilde{y} \rangle_{\mathbb{R}^n})+i(-\langle y,\tilde{x} \rangle_{\mathbb{R}^n} + \langle x,\tilde{y} \rangle_{\mathbb{R}^n})
			\end{split}
		\end{equation}
	\end{attention}
		\item Ist $T \in \End_{\mathbb{R}}(\mathbb{R}^n)$ normal, so ist auch $\hat{T} \in \End_{\mathbb{C}}(\mathbb{C}^n)$ normal.
	\end{proofenum}
	Die hier gemachten Bemerkungen sind nicht auf den $\mathbb{C}^n$ / $\mathbb{R}^n$ begrenzt. Allgemein gilt: \\
	\begin{theorem}
		Sei $V$ ein euklidischer Vektorraum (Vektorraum mit Skalarprodukt über $\mathbb{R}$). Dann existiert ein komplexer Vektorraum $V_{\mathbb{C}}$, so dass (i)-(v) gelten. (Details: Übung)
	\end{theorem}
	Nach dieser Vorbereitung können wir den Spektralsatz für normale Endomorphismen über $\mathbb{K} = \mathbb{R}$ formulieren. 
	\newpage
	\begin{theorem}{(Spektralsatz für normale Endomorphismen,  $\mathbb{K} = \mathbb{R}$)}
		Sei $V$ ein euklidischer Vektorraum. Dann ist $T \in \mathcal{L}(V)$ genau dann normal, wenn es für $T$ eine Orthonormalbasis $B$ von $V$ gibt, bzgl. der $M_B(T)$ die Gestalt \\
		\begin{equation}
			M_B(T) =
			\begin{pmatrix}
				\lambda_1 & \dots & 0 &0 & \dots & 0 \\
				\vdots &\ddots & \vdots & \vdots & \ddots& \vdots\\
				0 & \dots & \lambda_r & 0 &\dots & 0 \\
				0 & \dots & 0 & \mathcal{Q}_1 & \dots & 0 \\
				\vdots & \ddots & \vdots & \vdots & \ddots & \vdots \\
				0 & \dots & 0 & 0 & \dots & \mathcal{Q}_r
			\end{pmatrix}	
		\end{equation}
		Dabei sind $\lambda_1,...,\lambda_r \in \mathbb{R}$ die reellen Eigenwerte von $T$. Jedes $\mathcal{Q}_j \in M(2,\mathbb{R})$ ist von der Gestalt
		\begin{equation}
			\mathcal{Q}_j = 
			\begin{pmatrix}
				\alpha_j & \beta_j \\
				- \beta_j & \alpha_j
			\end{pmatrix}
		\end{equation}
		Dabei sind $\mu_j = \alpha_j + i\beta_j $ genau die komplexen Eigenwerte von $\hat{T}$ (=komplexe Nullstellen des charakteristischen Polynoms von $T$).
	\end{theorem}
	\begin{proof}
		Wir orientieren uns am Beweis von Satz 7.8.1 mit entsprechenden Modifikationen. \\
		Zunächst nehmen wir an, dass $M_B(T)$ obige Gestalt hat. Dann ist wiederum nach Satz 7.4.2
		\begin{equation}
			M_B(T^*) =
			\begin{pmatrix}
				\bar{\lambda}_1 & \dots & 0 &0 & \dots & 0 \\
				\vdots &\ddots & \vdots & \vdots & \ddots& \vdots\\
				0 & \dots & \bar{\lambda}_r & 0 &\dots & 0 \\
				0 & \dots & 0 & \mathcal{Q}_1^* & \dots & 0 \\
				\vdots & \ddots & \vdots & \vdots & \ddots & \vdots \\
				0 & \dots & 0 & 0 & \dots & \mathcal{Q}_r^*
			\end{pmatrix}
		\end{equation}
		also gilt 
		\begin{equation}
			M_B(T^*T) =
			\begin{pmatrix}
				|\lambda_1|^2 & \dots & 0 &0 & \dots & 0 \\
				\vdots &\ddots & \vdots & \vdots & \ddots& \vdots\\
				0 & \dots & |\lambda_r|^2 & 0 &\dots & 0 \\
				0 & \dots & 0 & \mathcal{Q}_1^* \mathcal{Q}_1 & \dots & 0 \\
				\vdots & \ddots & \vdots & \vdots & \ddots & \vdots \\
				0 & \dots & 0 & 0 & \dots & \mathcal{Q}_r^* \mathcal{Q}_r
			\end{pmatrix}
		\end{equation}
		Wenn man sich eines der Kästchen anschaut sieht man
		\begin{equation}
			\begin{split}
				\mathcal{Q}_j^* \mathcal{Q}_j &= \begin{pmatrix}
					\alpha_j & -\beta_j \\
					\beta_j & \alpha_j
				\end{pmatrix} \cdot \begin{pmatrix}
					\alpha_j & \beta_j \\
					-\beta_j & \alpha_j
				\end{pmatrix} = \begin{pmatrix}
					\alpha_j^2 + \beta_j^2 & 0 \\
					0 & \alpha_j^2 + \beta_j^2
				\end{pmatrix} \\
				&= \mathcal{Q}_j \mathcal{Q}_j^* 
			\end{split}
		\end{equation}
		Folglich gilt 
		\begin{equation}
			M_B(T^*T) = M_B(TT^*)
		\end{equation}
		also ist $T$ normal. \\
		Wir berechnen das charakteristische Polynom zunächst für $\mathcal{Q}_j$: \\
		\begin{equation}
			\begin{split}
				\det\begin{pmatrix}
					X - \alpha_j & \beta_j \\
					-\beta_j & X- \alpha_j
				\end{pmatrix} &= X^2-2\alpha_jX+\alpha_j^2+\beta_j^2 \\
				&= X^2 - 2 Re(\mu_j)X+ |\mu_j|^2, \ \ \  \mu_j := \alpha_j + i \beta_j \\
				&= (X- \mu_j) (X- \bar{\mu}_j)
			\end{split}
		\end{equation}
		Das heißt das charakteristische Polynom von T ist gegeben durch
		\begin{equation}
			\chi_T(X) = \prod\limits_{j=1}^r (X-\lambda_j) \prod\limits_{j=1}^s (X-\mu_j)(X-\bar{\mu}_j)
		\end{equation}
		Dies beendet den Beweis der Richtung $\Leftarrow$. \\
		Zum Beweis der Umkehrung wenden wir Induktion über $n = \dim (V)$. Für $n=1$ ist jeder Endomorphismus normal und schon in Diagonalgestalt, es ist also nichts zu zeigen. \\
		Die Behauptung sei bewiesen für Dimensionen $<n$ und sei nun $n=\dim V$ und $T \in \End_{\mathbb{R}}(V)$ normal gegeben. Dann gibt es die folgenden zwei Fälle: \\
		\begin{itemize}
			\item \textbf{1. Fall:} $T$ besitzt einen reellen Eigenwert.\\
				Dann erfolgt der Beweis wie in Satz 7.8.1. Man nehme einen Eigenvektor $e_1$, $||e_1||=1, \ Te_1 = \lambda_1 e_1, \ \lambda_1 \in \mathbb{R}$ und wendet die Induktionsvoraussetzung auf den normalen Endomorphismus $T|_{\langle e_1 \rangle^{\perp}}$ an.
			\item \textbf{2. Fall:} $T$ besitzt keinen reellen Eigenwert. \\
			Nun betrachten wir $\hat{T} \in \End_{\mathbb{C}}(V_{\mathbb{C}})$ (Satz 7.8.2).
			$\hat{T}$ ist normal mit gleichem charakteristischem Polynom wie $T$ (Übung). Sei also $\mu = \alpha + i \beta, \ \beta \neq 0$ ein komplexer Eigenwert von $\hat{T}$ mit normiertem Eigenvektor $\xi \in V_{\mathbb{C}}$. Es ist nach Satz 7.8.2 
			\begin{equation}
				\hat{T}(\overline{\xi})=\overline{\hat{T}(\xi)}= \overline{\mu \cdot \xi} = \overline{\mu} \cdot \overline{\xi}
			\end{equation}
			und da $\mu \neq \bar{\mu}$ und $\hat{T}$ normal ist, folgt aus Satz 7.7.4, dass $\xi \perp \bar{\xi}$, also mit 
			\begin{equation}
				\begin{split}
					\xi &= \zeta + i \eta \\
					\zeta :&= \frac{1}{\sqrt{2}}(\xi + \bar{\xi}) \\
					\eta :&= \frac{1}{\sqrt{2}i}(\xi - \bar{\xi}) \in V \\
					||\zeta||^2&= \frac{1}{2} (||\xi||^2+||\bar{\xi}||^2) =1 \\
					||\eta ||^2 &= 1 \ analog \\
					\langle \zeta, \eta \rangle &= \frac{1}{2i}\langle \xi + \bar{\xi}, \xi - \bar{\xi} \rangle \\
					&= \frac{1}{2i}(||\xi||^2-||\bar{\xi}||^2) \\
					&=0
				\end{split}
			\end{equation}
			Übung: $||\xi||=||\bar{\xi}||$. \\
			Das heißt, die Vektoren $\zeta,\eta \in V$ sind orthonormal und spannen einen 2-dimensionalen Unterraum von V auf. \\
			Bezüglich der Basis $\langle \zeta, \eta \rangle$ dieses Unterraums hat $T$ die Matrix
			\begin{equation}
				\begin{split}
					T\zeta &= \hat{T}\left(\frac{1}{\sqrt{2}}(\xi + \bar{\xi})\right) \\
					&= \frac{1}{\sqrt{2}}((\alpha+i\beta)\xi+(\alpha-i\beta)\bar{\xi}) \\
					&= \alpha\zeta - \beta \eta \\
					T\eta &= \hat{T}\left(\frac{1}{\sqrt{2}i}(\xi-\bar{\xi})\right) \\
					&= \frac{1}{\sqrt{2}i}((\alpha+i\beta)\xi-(\alpha-i\beta)\bar{\xi}) \\
					&= \alpha\eta + \beta\zeta \\
					&also \ gilt \\
					M_{\langle\zeta,\eta \rangle}(T) &= 
					\begin{pmatrix}
						\alpha & \beta \\
						-\beta & \alpha
					\end{pmatrix}
				\end{split}
			\end{equation}
			Da $T$ normal ist, sieht man nun wie im Beweis von Satz 7.8.1, dass T auch den Unterraum $U = \langle \zeta,\eta \rangle^{\perp}$ invariant lässt und $T|_U$ ebenfalls normal ist. Nun wendet man die Induktionsvoraussetzung auf $U$ an und ist fertig
		\end{itemize}
	\end{proof}
	
	\newpage	
	
	\subsection{Orthogonale und unitäre Abbildungen}
	Wie immer seien $V,W,...$ $\mathbb{K}$-Vektorräume mit Skalarprodukt. \\
	\begin{definition}
		$U \in \Hom_{\mathbb{K}}(V,W)$ heißt \textbf{Isometrie}, falls $||Ux||=||x|| \ \forall_{x \in V}$.
		Ist zusätzlich $U$ surjektiv, so nennt man $U$ unitär.
	\end{definition}
	\begin{remark}
		\begin{proofenum}
			\item Der Unterschied zu Abschnitt 7 ist, dass $U$ a priori zwischen verschiedenen Vektorräumen abbildet. Wir sehen gleich, dass Definition 7.9.1 mit dem Unitaritätsbegriff aus Abschnitt 7 konsistent ist.
			\item Die Terminologie weicht hier von Kowalsky ab. Dort wird $U$ schon unitär genannt, wenn es isometrisch ist. Hier wird es anders gemacht, da klar zwischen den Begriffen Isometrie und Isomorphismus klar unterschieden werden sollte und der Begriff unitär  den Isomorphismen von Vektorräumen mit Skalarprodukt vorbehalten werden sollte.
			\item Ist der Grundkörper $ \mathbb{K}=\mathbb{R}$, so spricht man auch von Orthogonalen Abbildungen.
			\item Aus Def. 9.1 folgt unmittelbar, das Isometrien injektiv sind. 
			\item Für Isometrien gilt die Äquivalenz $(i) \Leftrightarrow (ii) \Leftrightarrow (iii)$ des Satzes 7.7.5.
		\end{proofenum}
	\end{remark}
	\begin{theorem}
		Ist $U \in \Hom_{\mathbb{K}}(V,W)$ unitär im Sinne von Definition 7.9.1, so ist $U$ bijektiv, $U^*$ existiert und ist gleich $U^{-1}$ und es gelten $U^*U=Id_{V}, UU^*= Id_W$. \\
		Folglich ist Definition 7.9.1 mit Definition 7.1.2 konsistent.
	\end{theorem}
	\begin{proof}
		Die Bijektivität folgt aus der Definition. Wie oben bemerkt, gelte (i)-(iii) aus Satz 7.7.5. Also gilt für $x \in V, y \in W$
		\begin{equation}
			\langle Ux,y \rangle = \langle Ux,U(U^{-1}y) \rangle = \langle x, U^{-1}y \rangle
		\end{equation}
		somit gilt $U^{-1} = U^*$
	\end{proof}
	\begin{theorem}
			Sei $U \in \End_{\mathbb{K}}(V)$unitär. Dann ist $U$ normal und sämtliche komplexe Eigenwerte haben den Betrag 1.\\
			Sei umgekehrt $\dim V<\infty$ und $T \in \End_{\mathbb{K}}(V)$ normal mit sämtlichen komplexen Eigenwerten von Betrag 1, so ist $T$ schon unitär.			
	\end{theorem}
	\begin{proof}
		$\Rightarrow$: \\
		Sei $U$ unitär. Dann ist $U^*U=Id=UU^*$, also ist $U$ normal. Ist $Ux=\lambda x, x \neq 0$ ein Eigenvektor mit Eigenwert $\lambda$, so folgt
		\begin{equation}
			0 \neq ||x||=||Ux||=||\lambda x||=|\lambda|\cdot ||x||
		\end{equation}
		also $|\lambda|=1$. \\
		$\Leftarrow$: \\
		Sei nun $T \in \End_{\mathbb{K}}(V)$ normal mit $\dim V < \infty$, so dass alle Eigenwerte Betrag 1 haben. \\
		Ist $\mathbb{K=\mathbb{R}}$, so gehen wir zu $\hat{T} \in \End_{\mathbb{C}}(V_{\mathbb{C}})$ über. Die Eigenwerte von $\hat{T}$ sind sämtliche komplexe Eigenwerte von $T$, also von Betrag 1. Nach Satz 7.8.1 existiert eine Orthonormalbasis $B$ mit
		\begin{equation}
			M_B(\hat{T})=\begin{pmatrix}
				\lambda_1 & \dots & 0 \\
				\vdots & \ddots & \vdots \\
				0 & \dots & \lambda_n
			\end{pmatrix}
		\end{equation}
		folglich gilt
		\begin{equation}
			M_B(\hat{T}^*\hat{T})= \begin{pmatrix}
				|\lambda_1|^2 & \dots & 0 \\
				\vdots & \ddots & \vdots \\
				0 & \dots & |\lambda_n|^2
			\end{pmatrix} = Id
		\end{equation}
		Analog gilt $M_B(\hat{T}\hat{T}^*)=Id$.
		Also ist $\hat{T}$ unitär. \\		
		Es gilt aber
		\begin{equation}
			\hat{T^*T}= \hat{T^*}\hat{T}=\hat{T}^* \hat{T}=Id_{V_{\mathbb{C}}}=\hat{Id_V}
		\end{equation}
		woraus folgt
		\begin{equation}
			T^*T=TT^*=Id
		\end{equation}
	\end{proof}
	Aufgrund dieses Satzes sind die Spektralsätze des Abschnittes 7.8 auf unitäre Endomorphismen anwendbar. Wir gehen noch auf den reellen Fall ein: \\	
	Ist $U \in \End_{\mathbb{R}}(V), \dim V < \infty$ unitär, so besitzt $V$ nach Satz 7.8.3 eine Orthonormalbasis, sodass
	\begin{equation}
		M_B(U) = \begin{pmatrix}
			\lambda_1 & & & &0 & \\
			& \ddots& &  & & \\
			 & &\lambda_r & & & \\
			& & & \mathcal{Q}_1 & &  \\
			& & & & \ddots&  \\
			& 0& & & & \mathcal{Q}_s
		\end{pmatrix}
	\end{equation}
	Dabei ist $\lambda_j \in \mathbb{R}, |\lambda_j|=1$, also $\lambda_j \in \{ -1,1 \}$. \\
	Weiterhin ist
	\begin{equation}
		\mathcal{Q}_j = \begin{pmatrix}
			\alpha_j & \beta_j \\
			-\beta_j & \alpha_j
		\end{pmatrix}, \beta_j \neq 0
	\end{equation}		
	wobei $\mu_j = \alpha_j +i \beta_j$ ein komplexer Eigenwert von $\hat{U}$ ist. Nun ist
	\begin{equation}
		1 = |\mu_j|^2 = \alpha_j^2+\beta_j^2
	\end{equation}
	In der Analysis lernt man, dass es für $\alpha_j, \beta_j \in \mathbb{R}$ mit $\alpha_j^2+ \beta_j^2$ genau ein $\varphi_j \in (-\pi,\pi]$ gibt mit $\alpha_j = \cos (\varphi_j), \ \beta_j = \sin( \varphi_j)$. Damit können also die $\mathcal{Q}_j$ in der Form
	\begin{equation}
		\mathcal{Q}_j = \begin{pmatrix}
			\cos( \varphi_j) & \sin( \varphi_j) \\
			-\sin (\varphi_j )& \cos (\varphi_j)
		\end{pmatrix}
	\end{equation}
	geschrieben werden. \\
	\textbf{Die Polarzerlegung} \\
	In diesem Unterabschnitt gilt $\dim V < \infty$. \\
	Wir erinnern daran (§7.6), dass $A \in \End_{\mathbb{K}}(V)$ positiv (semi)definit heißt, falls $A = A^*$ und \\ 
	 $\forall_{x \in V} \langle Ax,x \rangle \geq 0$ (im Falle positiv definit: = 0 nur für $x=0$). \\
	Falls $\dim V<\infty$, so ist $A>0(\geq 0)$ genau dann, wenn sämtliche Eigenwerte $>0(\geq 0)$ sind. (Wir schreiben kurz $A >0 (\geq 0)$, falls $A=A^*$ und $A$ positiv (semi)definit ist.)
	\begin{theorem}{(Quadratwurzel)}
		Sei $\dim V <\infty$ und $A \in \End_{\mathbb{K}}(V)$ positiv semi-definit. \\
		Dann existiert genau ein $S \in \End_{\mathbb{K}}(V), S \geq 0$ mit $S^2=A$.
		(Schreibweise: $S = \sqrt{A}=A^{\frac{1}{2}}$) \\
		Die Eigenwerte von $S$ sind genau $\sqrt{\lambda}, \lambda \in spec(A)(\{\lambda \in \mathbb{K}| \exists v \in V: Tv = \lambda v \})$
	\end{theorem}
	\begin{proof}
		\textbf{Existenz:} \\	
		Sei $\{ e_1,...,e_n \}$ eine Orthonormalbasis von $V$ mit $Ae_j=\lambda_j e_j$, beachte $\lambda_j \geq 0$. \\
		Setze
		\begin{equation}
			Se_j := \sqrt{\lambda_j}e_j
		\end{equation}
		Das legt $S$ fest. Offenbar ist $S = S^* \geq 0$ und $S^2e_j = \lambda_je_j=Ae_j$, also $S^2 = A$. \\
		\textbf{Eindeutigkeit}: \\		
		Wir halten fest, dass auf den Eigenvektoren von $A$ zu den Eigenwerten $\lambda$ der Operator $S$ durch $Sx = \sqrt{\lambda}x$ operiert. \\
		Sei nun $\tilde{S} = \tilde{S}^* \geq 0, \tilde{S}^2=A$. Dann sei $\{ f_1,...,f_n \}$ eine Orthonormalbasis von $V$ mit $\tilde{S}f_j= \mu_jf_j, \mu_j \geq 0$. \\
		Es gilt $Af_j = \tilde{S}^2f_j = \mu_j^2f_j$, also ist $\mu_j^2$ ein Eigenwert von $A$.
		Dann folgt aber
		\begin{equation}
			Sf_j = \sqrt{\mu_j^2}f_j= \mu_jf_j = \tilde{S}f_j-
		\end{equation}
	\end{proof}
	\begin{exercise}
		Alternativbeweis mit Hilfe der Analysis.
	\end{exercise}
	\begin{definition}{\textbf{+ Satz}}
		Für $T \in \End_{\mathbb{K}}(V),\dim V < \infty$ ist $T^*T \geq 0$ und es gilt: \\
		$T^*T >0 \Leftrightarrow T$ invertierbar. \\
		Wir setzen $|T|:= \sqrt{T^*T}$
	\end{definition}
	\begin{proof}
		Da $\dim V < \infty$ ist injektiv $\Leftrightarrow$ bijektiv. \\
		\begin{equation}
			\langle T^*Tx,x \rangle = \langle Tx,Tx \rangle = ||Tx||^2 \geq 0
		\end{equation}
		Daraus folgt
		\begin{equation}
			x \in \ker T \Leftrightarrow \langle T^*Tx,x \rangle = 0
		\end{equation}
		Daraus folgt die Behauptung.
	\end{proof}
	\newpage
	\begin{theorem}{(Polarzerlegung)}
		Sei $\dim V < \infty$ und $T \in \End_{\mathbb{K}}(V)$. Dann ist $|T|$ der eindeutig bestimmte Operator $S \geq 0$ mit $||Tx||=||Sx|| \ \forall_{x \in V}$. \\
		Weiterhin existiert ein $U \in \End_{\mathbb{K}}(V)$ mit:
		\begin{itemize}
			\item $\ker U=\ker T$ 
			\item $U|_{\ker T^{\bot}}: \ker T^{\bot} \rightarrow \Img T$ unitär
			\item $T=U|T|$
		\end{itemize}
		Insbesondere gilt $U^*U|T|=|T|$, $U^*T=|T|$, $UU^*T=T$. \\
		Ist $T$ invertierbar, so auch $|T|$ und es gilt $U = T|T|^{-1}$
		\begin{remark}
			Man mache sich die Polarzerlegung in $\mathbb{C}$ klar. \\
			Wäre $T$ normal, so könnte man in den Eigenräumen die Polarzerlegung vornehmen. Im Allg. muss man sich etwas anderes überlegen.
		\end{remark}
	\end{theorem}
	\begin{proof}
		Für $|T|$ gilt sicherlich für $x \in V$:
		\begin{equation}
			\begin{split}
				||Tx||^2&=  \langle Tx, Tx \rangle = \langle T^*Tx,x \rangle \\
				&= \langle |T|^2x,x \rangle \\
				&= || \ |T|x \ ||^2
			\end{split}
		\end{equation}
		insbesondere auch $\ker T = \ker |T|$. \\
		Ist umgekehrt $S \geq 0$ und $||Sx||=||Tx|| \forall_{x \in V}$, so folgt:
		\begin{equation}
			\begin{split}
				\langle S^2x,x \rangle &= ||Sx||^2 = ||Tx||^2 \\
				&= \langle T^*Tx,x \rangle
			\end{split}
		\end{equation}
		Da $S^2,T^*T$ selbstadjungiert sind, folgt aus der Polarisierung $S^2=T^*T$, also $S = |T|$(mit Satz 7.9.5). \\
		Nach der Rang-Defekt-Formel ist 
		\begin{equation}
			\begin{split}
				\dim (\Img T) &= \dim V-\dim (\ker T) = \dim (\ker T^{\bot}) \\
				&= \dim (\ker |T|^{\bot}) \\
				&= \dim(\Img |T|)
			\end{split}
		\end{equation}
		Genauer gilt, da $|T|=|T|^*$: 
		\begin{equation}
			\Img |T| = \ker |T|^{\bot}
		\end{equation}
		Wir setzen für $y = |T|x \in \Img |T|$:
		\begin{equation}
			U_0y:=Tx
		\end{equation}
		$U_0$ ist wohldefiniert: ist $y = |T|x^{'}$, so ist $x-x^{'} \in \ker |T|=\ker T$, also auch $Tx = Tx^{'}$. \\
		$U_0$ ist isometrisch:
		\begin{equation}
			||U_0y||=||Tx|| = || \ |T|x \ || = ||y||
		\end{equation}
		Da $\dim (\Img |T|)=\dim (\Img T)$, ist $U_0$ unitär. \\
		Man setzt
		\begin{equation}
			U|_{\ker T}=0, U|_{\ker T^{\bot}}=U_0
		\end{equation}
		Es folgt nun für $x \in V$:
		\begin{equation}
			Tx = U_0|T|x=U|T|x
		\end{equation}
		also $T = U|T|$ \\		
		Rest: Übung.
	\end{proof}
	\newpage
	\section{Miscellanea}
	Es folgen Themen, für die es keine kanonische Einordnung gibt. 
	\subsection{Quotientenraum}
	Die Quotientenkonstruktion ist ein Standardverfahren zur Konstruktion neuer Vektorräume, welche nicht in offensichtlicher Weise $\mathbb{K}^n$'s sind. \\
	Im Folgenden sei V ein $\mathbb{K}$-Vektorraum und $U \subset V$ ein Unterraum. Wir betrachten die Nebenklassen $x+U, x \in V$. \\
	Etwas formaler sei für $x,y \in V$:
	\begin{equation}
		x \sim_{U} y: \Leftrightarrow x-y \in U
	\end{equation}
	Man prüft unmittelbar nach, dass $\sim_U$ eine Äquivalenzrelation auf V ist. Die Äquivalenzklasse von $x$ besteht aus:
	\begin{equation}
		[x]_{\sim_U} = \{ y \in V | x-y \in U \} = \{ x+u | u \in U \}=x+U
	\end{equation}
	\begin{definition}{(Und Satz)}
		$V/U:=V/\sim_U$ bezeichnet die Menge aller Äquivalenzklassen $mod \ U$. Man nennt $V/U$ den Quotientenraum nach $U$. Mittels der Verknüpfungen
		\begin{equation}
			\begin{split}
				(x+U)+(y+U) &:= (x+y)+U, \ x,y \in V \\
				\lambda \cdot (x+U) &:= (\lambda x)+U, \ \lambda \in \mathbb{K}, x \in V
			\end{split}
		\end{equation}
		wird $V/U$ zu einem $\mathbb{K}$-Vektorraum. \\
		Das neutrale Element ist $U = 0 + U$
	\end{definition}
	\begin{proof}
		Die standardmäßigen Details bleiben dem Leser erspart. Wir prüfen nur die Wohldefiniertheit der Verknüpfungen: \\
		Sind $x-x^{'}, y - y^{'} \in U$, so ist auch $(x+y)-(x^{'}+y^{'}) \in U$, also
		gilt 
		\begin{equation}
			(x+y) + U = (x^{'}+y^{'})+U
		\end{equation}
		Analog: Multiplikation
	\end{proof}
	\begin{theorem}{(Universelle Eigenschaft des Quotienten)}
		Sei $V$ ein $\mathbb{K}$-Vektorraum und $U \subset V$. \\
		Sei $\pi: V \rightarrow V/U, x \mapsto x+U$ die Quotientenabbildung. Dann gilt:
		\begin{proofenum}
			\item $\pi$ ist linear
			\item $\ker \pi = U$, $\pi$ ist surjektiv
			\item Falls $\dim V < \infty$, so ist $\dim V/U=\dim V-\dim U$
			\item Ist $V \xrightarrow{f} W$ eine lineare Abbildung mit $U \subset \ker F$, so gibt es genau eine lineare Abbildung $\hat{f}: V/U \rightarrow W$, so dass das folgende Diagramm kommutiert:
			\begin{center}
				\begin{tikzcd}
					V \arrow[d,"\pi", swap] \arrow[r, "f"] & W \\
					V/U \arrow[ur,"\hat{f}", swap]& \\
				\end{tikzcd}
			\end{center}
			Es gilt $\ker \hat{f}=\ker F/U$
		\end{proofenum}
	\end{theorem}
	\begin{proof}
		\begin{proofenum}
			\item klar
			\item Nach konstruktion ist $\pi$ surjektiv. \\
				\begin{equation}
					x \in \ker \pi \Leftrightarrow x+U = U \Leftrightarrow x \in U 
				\end{equation}
			\item
				Folgt aus der Rang-Defekt-Formel:
				\begin{equation}
					\begin{split}
						\dim V/U &= rg \pi \\
						&= \dim V-\dim (\ker \pi) \\
						&= \dim V-\dim U
					\end{split}
				\end{equation}
			\item Wir setzen $\hat{f}(x+U):=f(x)$. Ist $x+U = x^{'}+U$, so ist $x-x^{'} \in U$, also $f(x)=f(x^{'})$. Somit ist $\hat{f}$ wohldefiniert. Linearität folgt nun leicht. \\
			$x+U \in \ker \hat{f} \Leftrightarrow x \in \ker F \Leftrightarrow x+U \in (\ker f)/U$
			\begin{remark}
				$(\ker f)/U \subset V/U$ ist ein Unterraum.
			\end{remark}
		\end{proofenum}
	\end{proof}
	\begin{corollary}{2. Isomorphiesatz}
		Sei $W$ ein $\mathbb{K}$-Vektorraum und $U \subset V \subset W$ Unterräume. Dann gibt es einen kanonischen isomorphismus
		\begin{equation}
			\frac{W/U}{V/U} \xrightarrow{\cong} W/V
		\end{equation}
	\end{corollary}
	\begin{proof}
		Übung.
	\end{proof}
	
	\newpage	
	
	\subsection{Interne direkte Summe}
	Im Folgenden sei $V$ ein $\mathbb{K}$-Vektorraum.
	\begin{definition}{(und Satz)}
		Sei $(U_i)_{i \in I}$ eine Familie von Unterräumen von $V$. Dann bezeichnet
		\begin{equation}
			\sum\limits_{i \in I} U_i = \left\{ x \in V | \exists x_{i_1} \in U_{i_1},...,x_{i_r} \in U_{i_r}: x= x_{i_1}+...+x_{i_r}, r < \infty \right\}
		\end{equation}
		$\sum_{i \in I} U_i$ ist ein Unterraum von V. \\
		Ist $I = \{ 1,...,N \}$ endlich, so schreibt man auch $U_1 \oplus ... \oplus U_N$
	\end{definition}
	\begin{proof}
		Klar
	\end{proof}
	\begin{theorem}
		Für $(U_i)_{i \in I}$ sind äquivalent:
		\begin{proofenum}
			\item Jedes $x \in \sum U_i - \{ 0\}$ besitzt genau eine Darstellung $x = x_{i_1}+...+x_{i_r}$ mit eindeutig bestimmten $x_{i_j} \in U_{i_j} - \{ 0 \}$ und paarweise verschiedenen Indizes $i_1,...,i_r \in I$.
			\item
				Ist $x_{i_1}+...+x_{i_r}=0$ mit paarweise verschiedenen Indizes $i_1,...,i_r$ und $x_{i_j} \in U_{i_j}$, so ist schon $x_{i_1}=...=x_{i_r}=0$
			\item
				Für jedes $j \in I$ gilt:
				\begin{equation}
					U_j \cap \sum_{i \in I-\{ j\}}= \{0 \}
				\end{equation}
		\end{proofenum}
		Sind die äquivalenten Bedingungen erfülllt, so schreibt man $\bigoplus U_i$ und nennt es die direkte Summe. \\
		Ist die Summe direkt und sind $(B_i)_{i \in I}$ Basen von $U_i$, so ist $\bigcup_{i \in I} B_i$ eine Basis von $\bigoplus_{i \in I} U_i$. \\
		Insbesondere gilt, falls $I, \dim U_i < \infty$:
		\begin{equation}
			\dim \bigoplus_{i \in I} U_i = \sum_{i \in I} \dim U_i
		\end{equation}
	\end{theorem}
	\begin{proof}
		(i)$ \Rightarrow $(ii): \\
		 klar \\
		(ii) $\Rightarrow$ (iii): \\
		Sei
		\begin{equation}
			x = x_j=x_{i_1}+...+x_{i_r}
		\end{equation}
		mit $j \notin \{ i_1,...,i_r \}$. Dann ist $x_{i_1}+....+x_{i_r}-x_{i_j}=0$ und $(ii) \Rightarrow x_j =0=x_{i_1}=...=x_{i_r}$, also $x=0$. \\
		(iii)$\Rightarrow$(i); \\
		Sei 
		\begin{equation}
			x = x_{i_1}+...+x_{i_r}= y_{k_1}+....+y_{k_s}
		\end{equation}
		mit jeweils paarweise verschiedenen $i_1,...,i_r,k_1,k_s$ und $x_{i_j} \in U_{i_j}-\{ 0\}, y_{k_l} \in U_{k_l}-\{0 \}$. \\
		Wäre $i_1 \notin \{ k_1,...,k_s \}$, so wäre
		\begin{equation}
			\begin{split}
				U_{i_1} \ni x_{i_1} &= y_{k_1}+...+y_{k_s}-x_{i_2}-...-x_{i_r} \\
				& \in \sum_{l \neq i_1} U_l
			\end{split}
		\end{equation}
		und somit nach (iii) auch schon $x_{i_1} =0$. \\
		Widerspruch!\\
		Seien nun $B_i$ Basen von $U_i$. \\
		$B_i = \{ x_{ij} | j \in I_i \}$, $I_i$ eine Indexmenge. Ist $\xi = \xi_{i_1}+...+\xi_{i_n}$ mit $\xi_{i_j} \in U_{ij}$, so ist jedes $\xi_{i_j}$ Linearkombination von Elementen aus $B_{ij}$, also ist $\xi$ Linearkombination von Elementen aus $\bigcup_{i \in I} B_i$. \\
		Sei nun
		\begin{equation}
			\sum\limits_{endlich} \lambda_{i_l}x_{i_j}=0
		\end{equation}
		Dann ist wegen (ii) für jedes feste $i$ schon $\sum \lambda_{i_l}x_{i_l}=0$. Da aber $B_i$ linear unabhängig ist, folgt schon $\lambda_{i_l}=0, l \in I_i$.
		Die letzte Aussage ist nun klar.
	\end{proof}
	
	\newpage	
	
	\subsection{(Externe) direkte Summe und direktes Produkt}
	Im letzten Abschnitt waren die Räume $U_i$ Unterräume eines Raumes $V$, so dass $\bigoplus U_i$ als Unterraum von $V$ konstruiert werden konnte. Abstrakter kann man zu beliebigen Vektorräumen $U_i$ einen Raum $V$ und Einbettungen $U_i \hookrightarrow V$ so konstruieren, dass die direkte Summe $\bigoplus U_i$ gebildet werden kann. Allerdings gibt es für unendliche Indexmengen zwei Varianten. \\
	Seien $X_i, i \in I$ Vektorräume über $\mathbb{K}$. $I$ ist eine beliebige Indexmenge. Wir erinnern an das allgemeine kartesische Produkt
	\begin{equation}
		\prod\limits_{i \in I} X_i
	\end{equation}
	Dieses besteht aus allen Familien $(x_i)_{i \in I}$ mit $x_i \in X_i$. \\
	Etwas formaler sei $X$ irgendeine Menge mit $X_i \subset X$ für alle $i \in I$. Dann ist
	\begin{equation} 
		\prod\limits_{i \in I} X_i = \{ f: I \rightarrow X | \forall_{i \in I} f(i) \in X_i \}
	\end{equation}
	Wir führen auf $\prod_{i \in I} X_i$ Addition und Skalarmultiplikation ein durch
	\begin{equation}
		\begin{split}
			(f+g)(i) &:= f(i)+g(i), \ i \in I \\
			(\lambda f)(i) &:= \lambda \cdot f(i), \ i \in I 
		\end{split}
	\end{equation}
	für $f,g \in \prod_{i \in I}X_i, \lambda \in \mathbb{K}$. \\
	Man prüft unmittelbar nach:
	\begin{definition}{(Und Satz)}
		\begin{proofenum}
			\item 
				Sei $\prod_{i \in I} X_i$ mit $+, \cdot$ wie oben ein $\mathbb{K}$-Vektorraum. Man nennt es das "direkte Produkt" der Vektorräume $X_i, i \in I$.
			\item
				$\bigoplus_{i \in I} X_i :=\{ f \in \prod_{i \in I} X_i | f(i) \neq 0 , \ endlich \ viele \ i \}$ ist ein Unterraum von $\prod\limits_{i \in I}$. Man nennt ihn die "externe direkte Summe"
		\end{proofenum}
		\begin{remark}
			Falls $I$ endlich ist, so ist nach Defintion
			\begin{equation}
				\bigoplus_{i \in I} X_i = \prod_{i \in I} X_i
			\end{equation}
			Für unendliche $I$ hingegen gibt es sicherlich $f \in \prod_{i \in I} X_i$ mit $f(i) \neq 0$ für unendlich viele i (Auswahlaxiom). \\
			Ist $I = \{ i_1,...,i_n \}$ endlich, so schreibt man die Elemente von $\bigoplus X_i = \prod X_i$ auch als Tupel $\begin{pmatrix}
			x_{i_1}& x_{i_2} & \dots & x_{i_n}
			\end{pmatrix}$ oder Spalten $\begin{pmatrix}
			x_{i_1} \\
			x_{i_2} \\
			\vdots \\
			x_{i_n}
			\end{pmatrix}$.
		\end{remark}
	\end{definition}
	\begin{proof}
		\begin{proofenum}
			\item Einfache Übung. \\		
			\item Sind $f,g \in \bigoplus X_i, \lambda \in \mathbb{K}$, so ist $(f+\lambda \cdot g)$ höchstens $\neq 0$, wenn $f(i) \neq 0$ und $g(i) \neq 0$, also höchstens für endlich viele $i$.
		\end{proofenum}
	\end{proof}
	\begin{exercise}
		Universelle Eigenschaften von $\bigoplus_{i \in I} X_i$, $\prod_{i \in I} X_i$
	\end{exercise}
	Wir führen noch natürliche lineare Abbildungen im Zusammenhang mit direkter Summe und Produkt ein: \\
	für ein festes $\alpha \in I$ seien
	\begin{equation}
		\begin{split}
			\pi_{\alpha}: \prod_{i \in I} X_i &\rightarrow X_{\alpha} \\
			f &\mapsto f(\alpha) \\
			\kappa_{\alpha}: X_{\alpha} &\rightarrow \bigoplus_{i \in I} X_i \subset \prod_{i \in I} X_i \\
			x &\mapsto f_x, \\
			f_x(i) &= \begin{cases}
					0, i \neq \alpha \\
					x, i = \alpha
			\end{cases}
		\end{split}
	\end{equation}
	Man überlegt sich sich unmittelbar, dass 
	\begin{proofenum}	
		\item $\pi_{\alpha} \in \Hom_{\mathbb{K}}(\prod_{i \in I}X_i, X_{\alpha})$ surjektiv ist
		\item $\kappa_{\alpha} \in \Hom_{\mathbb{K}}(X_{\alpha}, \bigoplus_{i \in I} X_i)$ injektiv ist
		\item 
			\begin{equation}
				\begin{split}
				\pi_{\alpha} \circ \kappa_{\alpha}(x) &= x, \ x \in X_{\alpha} \\
				\pi_{\alpha} \circ \kappa_{\beta} &= 0, \ \alpha \neq \beta, \ \alpha, \beta \in I
				\end{split}
			\end{equation}
	\end{proofenum}
	Schließlich erhalten wir folgende Beziehung zwischen interner und externer direkter Summe:
	\begin{theorem}
		Die externe Summe $\bigoplus_{i \in I} X_i$ ist gleich der internen Summe $\bigoplus_{i \in I} \kappa_i X_i$ der Unterräume $\kappa_i X_i$.
	\end{theorem}
	\begin{proof}
		$\kappa_i$ ist injektiv, folglich ist $\kappa_i: X_i \rightarrow \kappa_i(X_i)$ ein kanonischer Isomorphismus auf den Unterraum
		\begin{equation}
			\kappa_i(X_i) = \left\{ f \in \bigoplus_{i \in I} X_i \left| f(j)=0, j \neq i \right. \right\}
		\end{equation}
		wir zeigen, dass die Familie $\kappa_i(X_i)$ die Bedingung (iii) des Satzes 8.2.2 erfüllt: \\
		Sei $j \in I$ und
		\begin{equation}
			f \in \kappa_j(X_j) \cap \sum_{i \in I-\{ j \}} \kappa_i(X_i)
		\end{equation}
		Dann ist einerseits $f(i) =0$ für $i \neq j$, andererseits $f = \sum_{ endl} \kappa_i(X_i)$ mit endlich vielen $x_i \in X_i, i \neq j$, also auch $f(j) = \sum (\kappa_i(X_i))(j)=0$, somit $f=0$. Ist nun $f \in ext - \bigoplus_{i \in I} X_i$, so ist nach Def. $f(i) \neq 0$ für höchstens endlich viele $i \in I$, etwa $i_1,...,i_n$. Also
		\begin{equation}
			f = \sum_{\alpha =1}^n \kappa_{i_{\alpha}}(f(\iota_{\alpha})) \in int- \bigoplus_{i \in I} \kappa_i(X_{i})
		\end{equation}
		andererseits ist ein Vektor
		\begin{equation}
			x = \sum_{\alpha =1}^n \kappa_{i_{\alpha}} (x_{\iota_{\alpha}}) \in int-\bigoplus_{i \in I} X_i
		\end{equation}
		sicherlich in $ext-\bigoplus_{i \in I}X_i$, denn
		\begin{equation}
			x(i) =0, i \notin \{ i_1,...,i_n \}
		\end{equation}
	\end{proof}
	
	\newpage	
	
	\subsection{Der Dualraum}
	\begin{definition}
		Sei $V$ ein $\mathbb{K}$-Vektorraum. Dann heißt
		\begin{equation}
			V^* := \Hom_{\mathbb{K}}(V, \mathbb{K})
		\end{equation}
		der Dualraum von $V$. \\
		Wir führen noch folgende Notation ein: \\
		Für $\varphi \in V^*, x \in V$ schreiben wir auch $\langle x, \varphi \rangle$ statt $\varphi(x)$.\\
		Die Notation erinnert an das Skalarprodukt und das ist durchaus gewollt. Für Formalisten: $\langle x, \varphi \rangle_{V,V^*}$
	\end{definition}
	\begin{theorem}
		Sei $V$ ein $ \mathbb{K}$-Vektorraum.
		\begin{proofenum}
			\item 
				Zu $x \in V-\{ 0 \}$ und $\lambda \in \mathbb{K}$ gibt es ein $\varphi \in V^*$ mit $\langle x, \varphi \rangle = \lambda$. Insbesondere ist $x \in V$ schon dann $0$, wenn $\langle x, \varphi \rangle=0$ für alle $\varphi \in V^*$
			\item 
				Ist $\dim(V) < \infty$ und $B = \{ e_1,...,e_n \}$ eine Basis von $V$, so ist durch
				\begin{equation}
					\langle e_i, e_j^* \rangle = \delta_{ij}
				\end{equation}
				eine Basis $B^* = \{ e_1^*,...,e_n^* \}$ von $V$ bestimmt. Man nennt $B^*$ die zu $B$ duale Basis.
		\end{proofenum}
	\end{theorem}
	\begin{proof}
		\begin{proofenum}
			\item
				Wir benötigen hierzu den allgemeinen Existenzsatz für Basen. Sei also $(e_i)_{i \in I}$ eine Basis von $V$ mit $e_{i_0}=x$. Dann besitzt jedes $y \in V$ eine eindeutige Darstellung
				\begin{equation}
					y = \sum_{endl} \alpha_i \cdot e_i
				\end{equation}
				Setze
				\begin{equation}
					\varphi(y):= \lambda \cdot a_{i_0}
				\end{equation}
				Dann besitzt $\varphi$ die gewünschte Eigenschaft.
			\item
				Lineare Abbildungen können auf Basen beliebig vorgegeben werden und dies legt sie auch fest, also ist $e_j^*$ durch (8.4.2) wohldefiniert. \\
				Ist $\varphi \in V^*$, so folgt für $x = \sum_{i=1}^n \lambda_i e_i$
				\begin{equation}
					\begin{split}
						\langle x, \varphi \rangle &= \sum_{i=0}^n \lambda_i \varphi(e_j) \\
						&= \sum_{i =1}^n \sum_{j=1}^n \lambda_i \varphi(e_j) \langle e_i, e_j^* \rangle \\
						&= \left\langle \sum_{i =1}^n \lambda_i e_i, \sum_{j =1}^n \varphi(e_j) e_j^* \right\rangle
					\end{split}
				\end{equation}
				Also folgt $\varphi = \sum_{j =1}^n \varphi(e_j) e_j^*$. Aus (8.4.2) folgt aber auch die lineare Unabhängigkeit von $B^*$.
		\end{proofenum}
	\end{proof}
	Für $\dim V < \infty$ ist also $V$ isomorph zu $V^*$. Man beachte, dass die Wahl eines Isomorphismus von der Wahl einer Basis abhängt. Es gibt keinen kanonischen Isomorphismus $V \rightarrow V^*$. \\
	Weiterhin wird für $\dim V = \infty$ das System $B^*$ zwar linear unabhängig sein, es wird $V^*$ jedoch nicht aufspannen. Genauer gilt:
	\begin{theorem}
		Sei $V$ ein $\mathbb{K}$-Vektorraum. Dann ist
		\begin{equation}
			\begin{split}
				\kappa: V &\rightarrow V^{**} \\
				\langle \varphi, \kappa(x) \rangle_{V^*,V^{**}} &:= \langle x, \varphi \rangle_{V, V^*}
			\end{split}
		\end{equation}
		eine (kanonische) injektive lineare Abbildung von $V$ in den Bidualraum $V^{**}:= (V^*)^*$. \\
		$\kappa$ ist genau dann ein Isomorphismus, wenn $\dim V < \infty$.
	\end{theorem}
	\begin{proof}
		Für jedes $x \in V$ ist sicherlich $V^* \ni \varphi \mapsto \langle x, \varphi \rangle$ eine Linearform (lineare Abbildung nach $\mathbb{K}$) auf $V^*$. Weiterhin ist für $x, y \in V, \lambda \in \mathbb{K}$
		\begin{equation}
			\begin{split}
				\langle \varphi, \kappa(\lambda x+y ) \rangle &= \langle \lambda x + y, \varphi \rangle =... \\
				&= \langle \varphi, \lambda \kappa(x) + \kappa(y) \rangle
			\end{split}
		\end{equation}
		also $\kappa \in \Hom_{\mathbb{K}}(V, V^{**})$. \\
		Ist $\kappa(x)=0$, so folgt $\langle x, \varphi \rangle =0$ für alle $\varphi \in V^*$ und Satz 8.4.2 impliziert $x=0$, also ist $\kappa$ injektiv.
	\end{proof}
	\begin{remark}
		Der Dualraum des $\mathbb{K}^n$ besitzt eine natürliche Beschreibung als Zeilenvektoren: wie üblich sei
		\begin{equation}	
			\mathbb{K}^n = \left\{  \begin{pmatrix}
				x_1 \\
				\vdots \\
				x_n
\end{pmatrix} | x_j \in \mathbb{K}	 \right\} = \mathcal{M}(n \times 1, \mathbb{K})
		\end{equation}
		die Menge der Spaltenvektore, bzw. der $n \times 1$-Matrizen mit der kanonischen Basis $e_j = \begin{pmatrix}
			0 \\
			\vdots \\
			1 \\
			0 \\
			\vdots \\
			0
		\end{pmatrix}$.\\
		 Betrachte $\varphi \in (\mathbb{K}^n)^*$. Dann ist für $x = \begin{pmatrix}
		 	x_1 \\
		 	\vdots \\
		 	x_n
		 \end{pmatrix} \in \mathbb{K}^n$
		 \begin{equation}
		 	\varphi(x) = \sum_{j = 1}^n x_j \varphi(e_j)= \begin{pmatrix}
		 		\varphi(e_1), \dots, \varphi(e_n)
		 	\end{pmatrix} \cdot \begin{pmatrix}
		 		x_1 \\
		 		\vdots \\
		 		x_n
		 	\end{pmatrix}
		 \end{equation}
		 Dabei haben wir stillschweigend die $1 \times 1$-Matrix ganz rechts mit ihrem Eintrag identifiziert; Puristen hätten auch die beiden linken Terme zusätzlich noch in Klammern setzen können. \\
		 \textbf{Fazit}: Der Dualraum des $\mathbb{K}^n$ ist in natürlicher Weise isomorph zum Raum $\mathcal{M}(1 \times n, \mathbb{K})$ der Zeilenvektoren.
	\end{remark}
	\begin{exercise}
		Hieraus erhält man einen weiteren abstrakten Beweis der Gleichheit von Zeilenrang und Spaltenrang einer Matrix. (vgl. Fischer Kap. 6)
	\end{exercise} 
	\textbf{Der Dualraum eines Vektorraums mit Skalarprodukt} \\
	Die Dualitätstheorie nimmt eine neue Wendung, falls $K = \mathbb{K} \in \{ \mathbb{R}, \mathbb{C} \}$ und $V$ mit einem Skalarprodukt versehbar ist.
	\begin{theorem}{\textbf{(und Definition)}}
		Sei $V$ ein Vektorraum mit Skalarprodukt. Dann ist 
		\begin{equation}
			\begin{split}
				b : V &\rightarrow V^* \\
				x &\mapsto \langle x, - \rangle
			\end{split} 
		\end{equation}
		eine injektive (anti)lineare Abbildung. D.h. falls $\mathbb{K}=\mathbb{R}$,so ist $b$ linear, falls $\mathbb{K}=\mathbb{C}$, so ist $b$ antilinear. \\
		Ist $\dim V < \infty$, so ist $b$ sogar ein Isomorphismus, d.h endlich-dimensionale Vektorräume mit Skalarprodukt sind \textbf{kanonisch} antiisomorph zu ihrem Dualraum.
	\end{theorem}
	\begin{remark}
		\begin{enumerate}
			\item 	
				Wir wissen schon, dass Vektorräume gleicher endlicher Dimension stets abstrakt isomorph sind. Deshalb ist es wesentlich, dass im obigen Satz die Abbildung natürlich (kanonisch) gegeben ist.
			\item 	
				Die letzte Aussage des Satzes nennt man auch den Darstellungssatz von Riesz.
		\end{enumerate}
	\end{remark}
	\begin{proof}
		Das Skalarprodukt ist linear im 2. Argument. Deshalb ist $x^b = \langle x,-\rangle$ für $x \in V$ eine Linearform auf $V$. Die Antilinearität der Zuordnung $x \mapsto x^b$ folgt aus der Antilinearität des Skalarprodukts im 1. Argument. \\
		Zum Beweis der Injektivität von $b$ sei $x^b = 0$. Dann ist also insbesondere $0 = x^b(x) = \langle x,x \rangle = ||x||^2$, also $x = 0$. Schließlich sei $\dim V < \infty$ und $\{ e_1,...,e_n \}$ eine Orthonormalbasis von $V$. Dann gilt für $\{ e_1^b,...,e_n^b \}$ aber
		\begin{equation}
			e_i^b(e_j) = \langle e_i, e_j \rangle = \delta_{ij}
		\end{equation}
		also folgt aus Satz 4.2.2, dass $\{ e_1^b,...,e_n^b \}$ die zu $\{ e_1,...,e_n \}$ duale Basis von $V^*$ ist. Insbesondere ist $b$ surjektiv.
	\end{proof}
	\textbf{Der Antiraum eines $\mathbb{C}$-Vektorraums} \\
	Die komplex antilinearen Abbildung verlieren ihre Sonderrolle, wenn man stattdessen einen $\mathbb{C}$-Vektorraum $V$ seinen Antiraum $\bar{V}$ an die Seite stellt. Dies stellt in gewisser Weise die Abstraktion der komponentenweisen Konjugation von Spaltenvektoren dar. \\
	Als additive abelsche Gruppe sei $\bar{V} = V$. Allerdings setzen wir für $\lambda \in \mathbb{C}, x \in V: \lambda \bar{\cdot} x = \bar{\lambda} \cdot x$. Für den Moment dient die Notation $\bar{\cdot}$ der Unterscheidung der beiden skalaren Multiplikationen. Man überprüft unmittelbar, dass $(\bar{V}, +, \bar{\cdot})$ ein $\mathbb{C}$-Vektorraum ist. Weiterhin ist $\langle x,y \rangle_{\bar{V}}:= \langle y,x \rangle_V$(Reihenfolge!) ein Skalarprodukt auf $\bar{V}$. \\
	Schließlich ist $id : V \rightarrow \bar{V}$ ein kanonischer antilinearer Isomorphismus. Das Wort "kanonisch" ist hier wesentlich, denn falls $\dim V < \infty$, so sind $V, \bar{V}$ auch (nicht-kanonisch) $\mathbb{C}$-isomorph. \\
	Nach dieser Vorrede können wir die Abbildung $b$ für einen Vektorraum mit Skalarprodukt nochmals betrachten: \\
	betrachtet man sie auf $\bar{V}$, so erhält man eine lineare Abbildung
	\begin{center}
		\begin{tikzcd}
			\bar{V} \arrow[rr, "\bar{\cdot}"  "antilinear"] \arrow[rrrr, bend right, "linear"]  &  & V \arrow[rr, "b" "antilinear"] & & V^* \\
			\bar{x} \arrow[rr, mapsto] & & x \arrow[rr, mapsto] & &x^b
		\end{tikzcd}
	\end{center}
	Das heißt, falls $\dim V < \infty$, so ist $\bar{V}$ kanonisch isomorph zu $V^*$. Alternativ ist die Abbildung $x \mapsto \langle -,x \rangle$ ein Isomorphismus von $V$ auf $\bar{V}^* = \bar{V^*}$
	
	\newpage	
	
	\subsection{Duale Abbildungen}
	Schon bei der Diskussion der Vektorräume mit Skalarprodukt tauchte die Frage auf, ob sich der Begriff der adjungierten Abbildung nicht auf allgemeine Räume ausdehnen lässt. In der Tat ist dies in gewisser Weise möglich. Dies und den Zusammenhang mit Vektorräumen mit Skalarprodukt wollen wir hier diskutieren. \\
	\begin{theorem}{\textbf{(und Definition)}}
		Seien $V,W$ $\mathbb{K}$-Vektorräume und sei $T \in \Hom_{\mathbb{K}}(V,W)$. Dann erklärt man $T^* \in \Hom_{\mathbb{K}}(W^*,V^*)$ durch
		\begin{equation}
			\langle V, T^* \varphi \rangle_{V, V^*} := \langle Tv, \varphi \rangle_{W, W^*}
		\end{equation}
		Man nennt $T^*$ auch die zu $T$ duale Abbildung. Die Zuordnung $Ad: \Hom_{\mathbb{K}}(V,W) \rightarrow \Hom_{\mathbb{K}}(W^*,V^*)$ ist linear.
	\end{theorem}
	\begin{proof}
		$T^*$ ist durch (8.5.1) wohldefiniert. Die Linearität von $T^*$ ebenso wie die von $Ad$ ergeben sich ebenfalls aus (8.5.1).
	\end{proof}
	Analog zum orthogonalen Komplement setzt man für $U \subset V$ den $U$-Annulator
	\begin{equation}
		U^o := \{ \varphi \in V^* | \forall_{x \in U} \langle x, \varphi \rangle = 0 \}
	\end{equation}
	Entsprechend gilt:
	\begin{theorem}
		Sei $T \in \Hom_{\mathbb{K}}(V, W)$ und $T^* \in \Hom_{\mathbb{K}}(W^*,V^*)$ die duale Abbildung. Dann gelten
		\begin{proofenum}
			\item 	
				$\ker T^* = (\Img T)^o$, $\ker T = (\Img T)^o \cap V$
			\item 
				$T^*$ injektiv $\Leftrightarrow$ $T$ surjektiv
			\item 
				$T^*$ surjektiv $\Leftrightarrow$ $T$ injektiv
			\item 
				Ist $\dim V < \infty$ oder $\dim W < \infty$, so ist $\rg T^* = \rg T$
		\end{proofenum}
	\end{theorem}
	\begin{proof}
		\begin{proofenum}
			\item 	
				\begin{equation}
					\begin{split}
						\varphi \in \ker T^* & \Leftrightarrow \forall_{x \in V} \langle x, T^* \varphi \rangle = 0 \\
						& \Leftrightarrow \forall_{x \in V} \langle Tx, \varphi \rangle = 0 \\
						& \Leftrightarrow \varphi \in T(V)^o
					\end{split}
				\end{equation}
				Analog gilt
				\begin{equation}
					\begin{split}
						x \in \ker T^* &\Leftrightarrow \forall_{\varphi \in V^*} \langle Tx, \varphi \rangle = 0 \\
						&\Leftrightarrow \forall_{\varphi \in V^*} \langle x, T^* \varphi \rangle = 0 \\
						&\Leftrightarrow x \in (T^* \varphi)^o \cap V
					\end{split}
				\end{equation}
				Beachte, dass $(T^* \varphi)^o \subset V^{**}$ und nach Satz 8.4.3 $V \subset V^**$
			\item 
				$T^*$ injektiv $\Leftrightarrow$ $\left( \forall_x \langle x, T^* \varphi \rangle = 0 \Rightarrow \varphi = 0 \right)$ $\Leftrightarrow$ $\left( \forall_x \langle Tx, \varphi \rangle = 0 \Rightarrow \varphi = 0 \right)$. \\
				Ist $T$ surjektiv, so ist die letzte Bedingung sicher erfüllt. Ist hingegen $T$ nicht surjektiv, so ist $W/T(V) \neq 0$. Wähle $\bar{\varphi} \in (W/T(V))^* -  \{ 0 \}$. Dann ist
				\begin{center}
					\begin{tikzcd}
						W \arrow[dr, bend left, "\varphi"] \arrow[d] & \\
						W/T(V) \arrow[r, "\bar{\varphi}"] & \mathbb{K}
					\end{tikzcd}
				\end{center}
				eine nichttriviale Linearform auf $W$, welche auf $T(V)$ verschwindet, also kann (5.2.1) nicht gelten. \\
				Insgesamt ist (5.2.1) äquivalent zur Surjektivität von $T$ und (ii) ist bewiesen. \\
			\item
				$\Rightarrow:$ Sei $T^*$ surjektiv. Dann liefert (i)
				\begin{equation}
					\ker T = (\Img T^*)^o \cap V = (V^*)^o \cap V = \{ 0 \} 
				\end{equation}
				$\Leftarrow$: Sei $T$ injektiv und $(x_i)_{i \in I}$ eine Basis von $V$. Dann ist $(Tx_i)_{i \in I}$ ein linear unabhängiges System in $W$. Ergänze dies zu einer Basis $(y_j)_{j \in J}$ von $W, J \supset I, y_i = Tx_i$ für $i \in I$. Ist nun $\varphi \in V^*$ gegeben, so wähle $\Psi \in W^*$ mit
				\begin{equation}
					\begin{split}
						\Psi(Tx_i) &= \varphi (x_i) \\
						\Psi (y_j) &= 0, j \in J - I
					\end{split}
				\end{equation}
				Beachte, dass Lineare Abbildungen auf Basen frei vorgegeben werden können. Es ist nun für $x = \sum_{endl.} \lambda_i x_i \in V$
				\begin{equation}
					\begin{split}
						\langle x, T^* \Psi \rangle &= \langle Tx, \Psi \rangle \\
						&= \sum_{endl} \lambda_i \langle Tx_i, \Psi \rangle \\
						&= \sum_{endl} \lambda_i \langle x_i, \varphi \rangle \\
						&= \langle x, \varphi \rangle
					\end{split}
				\end{equation}
				also $T^* \Psi = \varphi$. Da $\varphi \in V^*$ beliebig war, folgt die Surjektivität von $T^*$.
			\item 
				Sei etwa $\dim W < \infty$. 
				\begin{exercise}
					$\dim V < \infty$
				\end{exercise}
				\begin{equation}
					\begin{split}
						\Rightarrow \rg T^* &= \dim W - \dim \ker T^* \\
						&= \dim W- \dim (\Img T)^o \ \ \ (i) \\
						&= \dim W - (\dim W^* - \rg T) \\
						&= \rg T
					\end{split}
				\end{equation}
				Dabei wurde $\dim W = \dim W^*$ sowie folgendes Resultat verwendet
				\begin{exercise}
					Sei $X$ ein $\mathbb{K}$-Vektorraum, $\dim X < \infty$ und $Y \subset X$ ein Unterraum. Dann ist $\dim Y^o = \dim X - \dim Y$
				\end{exercise}
		\end{proofenum}
	\end{proof}
	Im Folgenden diskutieren wir den Zusammenhang zwischen dualen Abbildungen und Matrizen:
	\begin{theorem}
		Seien $V,W$ $\mathbb{K}$-Vektorräume endlicher Dimension sowie $T \in \Hom_{\mathbb{K}}(V,W)$. Seien $A,B$ Basen von $U$ bzw. $W$ mit zugeordneten dualen Basen $A^*, B^*$. Dann gilt
		\begin{equation}
			M_{B^*,A^*}(T^*) = M_{A,B}(T)^t
		\end{equation}		 
	\end{theorem}
	\begin{proof}
		Seien $A = \{ a_1,...,a_n \}, B = \{ b_1,....,b_m \}$, \\
		 $M_{A,B} =: (t_{ij})_{ 1 \geq i \geq m, \ \ 1 \geq j \geq n}$, \\ 
		 $T a_j = \sum_{i = 1}^m t_{ij} b_i$,\\
		 $T^* b_j^* = \sum_{i = 1}^n \bar{t_{ij}} a_i^*$.
		\begin{center}
			\begin{tikzcd}
			\langle a_k, T^* b_j^* \rangle \arrow[r, equal] \arrow[d, equal] & \sum_{i=1}^n \bar{t_{ij}} \langle a_k, a_i^* \rangle \arrow[r, equal] & \bar{t_{kj}} \arrow[d, equal, "\textbf{!}"] \\
			\langle T a_k, b_j^* \rangle \arrow[r, equal] & \sum_{l = 1}^m t_{lk} \langle b_l, b_j^* \rangle \arrow[r, equal] & t_{jk}
			\end{tikzcd}
		\end{center}
	\end{proof}
	\textbf{Duale Abbildungen und adjungierte Abbildungen im Falle von Vektorräumen mit Skalarprodukt} \\
	Von nun an sind $V,W,...$ endlich-dimensionale $\mathbb{K}$-Vektorräume ($\mathbb{K} \in \{ \mathbb{R}, \mathbb{C} \}$) mit Skalarprodukt. Hier müssen wir notationell etwas pedantisch sein: \\
	$\langle -|- \rangle_V$ bezeichnet Skalarprodukte, $\langle -,- \rangle_{V,V^*}$ die Dualität. Weiterhin bezeichnet $T^t$ die duale (transponierte) Abbildung und $T^{ad}$ die adjungierte Abbildung. $b: V \rightarrow V^*$ bezeichnet den Isomorphismus aus Satz 8.4.4 und $\phi: V^* \rightarrow V$ bezeichne dessen Inverses. Beachte, dass für $\mathbb{K}=\mathbb{C}$ die Abbildung $b,\phi$ antilinear sind, bzw. linear, wenn man zu $\bar{V}$ übergeht. \\
	Für $x,y \in V$ gilt nun
	\begin{equation}
		\langle y, x^b \rangle_{V,V^*} = x^b(y) = \langle x | y \rangle_V \overset{!}{=} \langle y | x \rangle_{\bar{V}}
	\end{equation}
	M. a. W. durch $b$ geht das Skalarprodukt auf $\bar{V}$ in die Dualität von $V, V^*$ über. \\
	Sei nun $T \in \Hom_{\mathbb{K}}(V,W)$. Dann gilt für $x \in V, y \in W$: 
	\begin{equation}
		\begin{split}
			\langle x, T^t y^b \rangle &= \langle x, (T^t y^b)^{\phi b} \rangle_{V,V^*} \\
			&= \langle (T^t y^b)^{\phi} | x \rangle_V
		\end{split}
	\end{equation}
	andererseits
	\begin{equation}
		= \langle Tx, y^b \rangle_{W, W^*} = \langle y | Tx \rangle_W
	\end{equation}
	Liest man dies von ganz rechts rückwärts, so sehen wir, dass $T^{ad} = \phi \circ T^t \circ b$. \\
	Auf den ersten Blick mag dies erstaunen, denn $T \mapsto T^t$ ist linear, während $T \mapsto T^{ad}$ antilinear ist. Es hat jedoch alles seine Richtigkeit, denn wegen der Antilinearität von $\phi, b$ erhalten wir für $x \in W, \lambda \in \mathbb{K}$
	\begin{equation}
		\begin{split}
			(\lambda T^{ad})(x) &= \left( (\lambda T )^t x^b \right)^{\phi} = (\lambda T^t x^b)^{\phi} \\
			&= \bar{\lambda}(T^t x^b)^{\phi} \\
			&= \bar{\lambda} T^{ad}(x)
		\end{split}
	\end{equation}
	
	\newpage	
	
	\subsection{Das Tensorprodukt}
	
	Eine weitere vielseitige Konstruktion von neuen Vektorräumen aus gegebenen ist das Tensorprodukt. \\
	Im Folgenden seien $X_i, Y_i, Z_i, i = 1,2...$ Vektorräume über dem Körper $\mathbb{K}$. Im Abschnitt über Determinanten hatte wir bereits multilineare Abbildungen berührt. Ebenso im Abschnitt pber Bilinearformen
	\begin{definition}
		Eine Abbildung $T: X_1 \times .... \times X_n \rightarrow Y$ heißt \textbf{n-fach multilinear}, wenn für jedes $i \in \{ 1,...,n \}$ und feste Vektoren $x_1,..., \hat{x_i},....,x_n$ die Abbildung
		\begin{equation}
			T(x_1,...,x_{i-1},x_{i+1},...,x_n) \in \Hom_{\mathbb{K}}(X_i,Y)
		\end{equation}
		Schreibweise: $\mathcal{L}(X_1,...,X_n;Y)$ bezeichnet die Menge der n-fach multilinearen Abbildungen $X_1 \times .... \times X_n \rightarrow Y$.
	\end{definition}
	\begin{example}
		\begin{proofenum} 
			\item 
				Ist $\dim V = n$, so ist die Determinantenfunktion
				\begin{equation}
					\det : V \times ... \times V \rightarrow \mathbb{K}
				\end{equation}
			 	n- fach multilinear.
			 \item 
			 	Eine Bilinearform ist 2-fach multilinear. 
		\end{proofenum}
	\end{example}
	\begin{remark}
		\begin{enumerate}
			\item 
				Mit der punktweisen Addition und skalaren Multiplikation wird $\mathcal{L}(X_1,...,X_n;Y)$ selbst zu einem Vektorraum über $\mathbb{K}$. Für $n=1$ gilt trivialerweise $\mathcal{L}(X_1;Y)=\Hom_{\mathbb{K}}(X,Y)$
			\item
				Für ein festes $1 \leq k \leq n$ ist
				\begin{equation*}
					\begin{split}
						\phi : \mathcal{L}(X_k, \mathcal{L}(X_1,...,\hat{X_k},...,X_n;Y)) &\rightarrow \mathcal{L}(X_1,...,X_n;Y) \\
						\phi(T)(x_1,...,x_n) := T(x_k)(x_1,...,\hat{x_k},....,x_n)
					\end{split}
				\end{equation*}
				ein Isomorphismus.
			\item
				Sind alle involvierten Vektorräume endlich-dimensional, so gilt
				\begin{equation}
					\dim \left( \mathcal{L}(X_1,...,X_n;Y) \right) = \left( \prod_{j = 1}^n \dim X_j \right) \cdot \dim Y
				\end{equation}
		\end{enumerate}
	\end{remark}
	\begin{definition}{\textbf{(Konstruktion des Tensorprodukts)}}
		Man starte mit dem (Riesen-)raum, welcher dadurch entsteht, dass wir die Elemente von $X_1 \times .... \times X_n$ als Indexmenge auffassen:
		\begin{equation}
			\bar{\zeta}:= \Abb (X_1, \times .... \times X_n, \mathbb{K})
		\end{equation}
		Bekanntlich ist für jede Menge $\Xi$ der Raum $\Abb (\Xi, \mathbb{K})$ ein $\mathbb{K}$-Vektorraum bezüglich punktweiser Addition und skalar Multiplikation. Man bezeichne für $x_i \in X_i$
		\begin{equation}
			\begin{split}
				&(x_1,...,x_n) \in \bar{\zeta}  \\
				&(x_1,...,x_n) \left( (y_1,...,y_n) \right)= \begin{cases}
						1, x_1 = y_1,....,x_n= y_n \\
						0, sonst
					\end{cases}
			\end{split}
		\end{equation}
		Dann ist $\bar{\zeta}$ der Raum aller (formalen) Linearkombinationen
		\begin{equation}
			\sum_{endl., \ 
			(x_1,...,x_n) \in X_1 \times .... \times X_n} \lambda_{(x_1,...,x_n)} (x_1,...,x_n)
		\end{equation}
		Sei $\mathcal{U}$ der Unterraum von $\bar{\zeta}$, welche aufgespannt wird von den Vektoren der Form
		\begin{equation}
			x = (x_1,...,x_{i-1}, \eta + \lambda \cdot \gamma, x_{i+1},....,x_n)-(x_1,....,x_{i-1},\eta,x_{i+1},....,x_n) - \lambda \cdot (x_1,....,x_{i-1}, \gamma,x_{i+1},....,x_n)
		\end{equation}
		Der Quotientenraum $\bar{\zeta}/\mathcal{U}$ heißt \textbf{Tensorprodukt} der Räume $X_1,...,X_n$ über $\mathbb{K}$. \\
		Schreibweise: $X_1 \otimes_{\mathbb{K}} .... \otimes_{\mathbb{K}} X_n $. \\
		Die kanonische Quotientenabbildung wird als
		\begin{equation}
			\begin{split}
				\pi : X_1 \times .... \times X_n &\rightarrow X_1 \otimes ... \otimes X_n \\
				(x_1,...,x_n) &\mapsto x_1 \otimes ... \otimes x_n
			\end{split}
		\end{equation}
	\end{definition}
	\begin{theorem}{(Universelle Eigenschaft des Tensorprodukts)}
		\begin{proofenum}
			\item 	
				$\pi$ ist n-fach multilinear. 
			\item
				Zu jeder n-fach multilinearen Abbildung $T \in \mathcal{L}(X_1,...,X_n;Y)$ gibt es genau eine lineare Abbildung $\hat{T} \in \mathcal{L}(X_1 \otimes ... \otimes X_n;Y)$, so dass das Diagramm
				\begin{center}
					\begin{tikzcd}
						X_1 \times ... \times X_n \arrow[d, "\pi" ,swap] \arrow[r, "T"] & Y \\
						X_1 \otimes ... \otimes X_n \arrow[ur, "\hat{T}",swap]
					\end{tikzcd}
				\end{center}
				kommutiert.
			\item 
				Ist $Z$ ein $\mathbb{K}$-Vektorraum mit $\bar{\pi} \in \mathcal{L}(X_1,...,X_n;Z)$, so dass (ii) sinngemäß gilt, wenn man $X_1 \otimes ... \otimes X_n$ durch $Z$ und $\pi$ durch $\bar{\pi}$ ersetzt, so ist $Z$ kanonisch isomorph zu $X_1 \otimes .... \otimes X_n$. Unter diesem Isomorphismus geht $\bar{\pi}$ in $\pi$ über.
		\end{proofenum}
	\end{theorem}
	\begin{proof}
		\begin{proofenum}
			\item 
				Ist aus der Definition von $\bar{\zeta}/\mathcal{U}$ offensichtlich.
			\item
				Zu gegebenem $T \in \mathcal{L}(X_1,...,X_n;Y)$ setzt man
				\begin{equation}
					\begin{split}
						\tilde{T} : \bar{\zeta} &\rightarrow Y \\
						(x_1,...,x_n) \mapsto T(x_1,...,x_n)
					\end{split}
				\end{equation}
				Beachte, dass für $x_j \in X_j$ der Vektor $(x_1,...,x_n)$ ein Basisvektor von $\bar{\zeta}$ ist. Daher ist $\tilde{T} \in \Hom_{\mathbb{K}}(\bar{\zeta}, Y)$. Weiterhin ist
				\begin{equation}
					\begin{split}
						\tilde{T} ( &(x_1,...,x_{i-1}, \eta + \lambda \cdot \gamma, x_{i+1},....,x_n)-(x_1,....,x_{i-1},\eta,x_{i+1},....,x_n) \\
						 &- \lambda \cdot (x_1,....,x_{i-1}, \gamma,x_{i+1},....,x_n) ) \\
						 &= 0
					\end{split}
				\end{equation}
				also gilt $\mathcal{U} \subset \ker \tilde{T}$. Folglich gibt es nach der Universellen Eigenschaft des Quotienten eine lineare Abbildung $\hat{T}: \bar{\zeta}/\mathcal{U} \rightarrow Y$, so dass das Diagramm
				\begin{center}
					\begin{tikzcd}
						\bar{\zeta} \arrow[r, "\tilde{T}"] \arrow[d] & Y \\
						X_1 \otimes ... \otimes X_n = \bar{\zeta}/\mathcal{U} \arrow[ur, "\hat{T}", swap]
					\end{tikzcd}
				\end{center}
				kommutiert. Für $x_j \in X_j$ gilt
				\begin{equation}
					\hat{T}(x_1 \otimes ... \otimes x_n) = \tilde{T}\left( (x_1,...,x_n) \right)= T(x_1,...,x_n)
				\end{equation}
			\item 
				Wir haben das kommutative Diagram
				\begin{center}
					\begin{tikzcd}
						X_1 \times ... \times X_n \arrow[rr, "\tilde{\pi}"] \arrow[ddd, "\pi"] & & Z \arrow[dddll, bend right, "\bar{\pi}"] \\
						& \\
						& \\
						X_1 \otimes ... \otimes X_n \arrow[uuurr, bend right, "\hat{\tilde{\pi}}", swap]
					\end{tikzcd}
				\end{center}
				Es folgt sofort
				\begin{equation}
					\begin{split}
						\left( \bar{\pi} \circ \hat{\tilde{\pi}} \right) (x_1 \otimes ... \otimes x_n) &= \left( \bar{\pi} \circ \tilde{\pi} \right)(x_1,...,x_n) \\
						&= \pi(x_1,...,x_n) \\
						&= x_1 \otimes ... \otimes x_n
					\end{split}
				\end{equation}
				Aus der Eindeutigkeit von $\bar{\pi}$ folgt die Surjektivität von $\bar{\pi}$. Also erhält man durch Vertauschen der Rollen von $\pi$ und $\tilde{\pi}$ auch $\hat{\tilde{\pi}} \circ \bar{\pi} = id$.
		\end{proofenum}
	\end{proof}
	Vektoren von $X_1 \otimes ... \otimes X_n$ haben also die Gestalt
		\begin{equation}
			\sum_{endl} x_1^{(i)} \otimes ... \otimes x_n^{(i)}
		\end{equation}
		mit $x_j^{(i)} \in X_j$. Allerdings ist eine solche Summenzerlegung im Allgemeinen nicht eindeutig! \\
		Allerdings hat man den
		\begin{theorem}
			Seien $B_1,...,B_n$ Basen von $X_1,...,X_n$. Dann ist
			\begin{equation}
				B:= \left\{ b_1 \otimes ... \otimes b_n | \forall_i b_i \in B_i \right\}
			\end{equation}
			eine Basis von $X_1 \otimes ... \otimes X_n$. \\
			Insbesondere ist im Falle endlicher Dimensionen
			\begin{equation}
				\dim X_1 \otimes ... \otimes X_n = \prod_{j=1}^n \dim X_j
			\end{equation}
		\end{theorem}
		\begin{proof}
			\begin{enumerate}
				\item 
					Es ist offensichtlich, dass span$(B) = X_1 \otimes ... \otimes X_n$.  \\
					Es genügt also, jeden solchen Vektor als Element von span$(B)$ zu identifizieren. Sei etwa \\
					\begin{equation}
						\begin{split}
							x_j &= \sum_{endl} \lambda_{ij} b_i \\
							\Rightarrow x_1 \otimes ... \otimes x_n &= \sum_{endl} \lambda_{i_1 1} \lambda_{i_2 2} \cdot ... \cdot \lambda_{i_n n} \ b_{i_1} \otimes ... \otimes b_{i_n} \in span(B)
						\end{split}
					\end{equation}
				\item
					Lineare Unabhängigkeit: Hierzu verwende man die universelle Eigenschaft. Um die Notation nicht zu überfrachten, beschränken wir uns auf $n = 2$ (der Rest folgt induktiv): \\
					Sei also $X_1 = X$, $X_2 = Y$ und $B_1 = \{ b_i | i \in I \}, B_2 = \{ c_j | j \in J \}$ mit Indexmengen $I,J$. Setze obsulet
					\begin{equation}
						\begin{split}
							\hat{b_i} &:= (b_i, 0) \in X \times Y = X \oplus Y \\
							\hat{c_j} &:= (0,c_i) 
						\end{split}
					\end{equation}
					Dann ist $\left\{  \hat{b_i} | i \in I \right\} \cup \left\{ \hat{c_j} | j \in J \right\}$ eine Basis von $X \oplus Y$. \\
					Man fixiere nun $i_0 \in I, j_0 \in J$: \\
					Betrachte
					\begin{equation}
						\begin{split}
							\lambda_{i_0,j_0}: X \times Y &\rightarrow \mathbb{K} \\
							(\sum \alpha_i b_i, \sum \beta_j c_j) &\mapsto \alpha_{i_0} \cdot  \beta_{i_0}
						\end{split}
					\end{equation}
					$\lambda_{i_0,j_0}$ ist bilinear, also gibt es nach der universellen Eigenschaft ein kommutatives Diagramm
					\begin{center}
						\begin{tikzcd}
							X \times Y \arrow[r, "\lambda_{i_0,j_0}"] \arrow[d] & \mathbb{K} \\
							X \otimes Y \arrow[ur, "\hat{\lambda_{i_0,j_0}}", swap]
						\end{tikzcd}
					\end{center}
					Sei nun
					\begin{equation}
						\begin{split}
							& \sum_{endl} \alpha_{k,l} b_k \otimes c_l = 0 \\
							\Rightarrow 0 &= \hat{\lambda_{i_0,j_0}}\left( \sum_{endl} \alpha_{k,l} b_k \otimes c_l \right) \\
							&= \sum_{endl} \alpha_{k,l} \lambda_{i_0,j_0}(b_k,c_l) \\
							& \overset{!}{=} \alpha_{i_0,j_0}
						\end{split}
					\end{equation}
					Da $i_0,j_0$ beliebig waren, folgt $\forall_{k,l} \alpha_{k,l} =0$ und damit die lineare Unabhängigkeit von $\left\{ b_i \otimes c_j \right\}$.
			\end{enumerate}
		\end{proof}
		\begin{example}
			\begin{proofenum}
				\item
					Komplexifizierung/Koeffizientenerweiterung: \\
					Sei $V$ ein $K$-Vektorraum und $L \supset K$ eine Körpererweiterung. Dann kann $L$ auch als $K$-Vektorraum aufgefasst werden.
					\begin{equation}
						V_L := L \otimes_{\mathbb{K}} V
					\end{equation}
					ist dann in natürlicher Weise ein $L$-Vektorraum. Im Fall $K = \mathbb{R}, L = \mathbb{C}$ entspricht dies der schon besprochenen Komplexifizierung. 
				\item 
					Seien $V,W$ $K$-Vektorräume endlicher Dimension
					\begin{equation}
						\begin{split}
							\Theta: V^* \times W &\rightarrow \Hom_{K}(V,W) \\
							(v^*,w) &\mapsto \Theta_{V^*,W}
						\end{split}
					\end{equation}
					mit $\Theta_{V^*,W}(v)=\langle v, v^* \rangle_{V,V^*}w$. \\
					$\Theta$ is bilinear, also gibt es ein kommutatives Diagramm
					\begin{center}
						\begin{tikzcd}
							V^* \times W \arrow[r, "\Theta"] \arrow[d] & \Hom_{K}(V,W) \\
							V^* \otimes W \arrow[ur, "\hat{\Theta}",swap]
						\end{tikzcd}
					\end{center}
					Mittels Basen prüft man nach, dass $\hat{\Theta}$ surjektiv ist. Da aber 
					\begin{equation}
						\dim \Hom_{K}(V,W) = \dim V \cdot \dim W = \dim (V^* \otimes W)
					\end{equation}
					muss $\hat{\Theta}$ ein Isomorphismus sein.
			\end{proofenum}
		\end{example}
		\begin{exercise}
			\begin{enumerate}
				\item 
					\begin{equation}
						X \otimes Y \cong Y \otimes X
					\end{equation}
				\item 
					Seien $X = X_1 \oplus X_2, Y = Y_1 \oplus Y_2$, so folgt
					\begin{equation}
						X \otimes Y = \left( X_1 \otimes Y_1 \right) \oplus \left( X_1 \otimes Y_2 \right) \oplus \left( X_2 \otimes Y_1 \right) \oplus \left( X_2 \otimes Y_2 \right)
					\end{equation}
					\begin{corollary}
												
						Ist $(b_i)_{i \in I}$ eine Basis von $X$, so hat jedes $\zeta \in X \otimes Y$ eine Darstellung
						\begin{equation}
							\zeta = \sum_{endl} b_i \otimes y_i
						\end{equation}
						mit eindeutig bestimmten $y_i \in Y$, wobei $y_i=0$ für alle, bis auf endlich viele $y_i$.
					\end{corollary}
			\end{enumerate}
		\end{exercise}
		
		\newpage		
		
		\subsection{Ein weniger naiver Blick auf Mengen}
		
		Die Existenz einer Basis eines beliebigen Vektorraums kann aus den bisher verwendeten Axiomen der naiven Mengenlehre nicht abgeleitet werden. Man benötigt dazu in ein oder anderer Form das Auswahlaxiom. Wir wollen dies hier nun etwas näher beleuchten. Naturgemäß wird es jetzt etwas formaler. Es wird daran erinnert, dass sämtliche mathematische Objekte letztlich Mengen sind.
		\begin{definition}{(Familie)}
			Eine Familie $(X_i)_{i \in I}$ von Mengen ist eine Abbildung $X_{\bullet}: I \rightarrow Z$. Dabei sind $I, Z$ Mengen.
		\end{definition}
		Diese Definition erfolgt zur Klarstellung, dass "Familien" keine außerhalb der Mengen stehende Objekte sind.
		Vielmehr ist es lediglich eine bequeme Schreibweise. Es steht weniger die Abbildung im Vordergrund, als die Kollektion der Mengen $X_i$. Allerdings könnte man alternativ $\mathcal{F} = \{ Y \in \mathcal{P}(Z) | \exists_i X_i = Y \}$ betrachten und dies gleichzeitig als Indexmenge verwenden.
		Das ist weniger intuitiv, aber logisch das gleiche.
		Wir sind nun in der Lage, das Auswahlaxiom zu formulieren
		\begin{axiom}{(Auswahlaxiom)(AC)}
			Ist $\forall_i X_i \neq \emptyset$, so ist $\prod_{i \in I} X_i \neq \emptyset$
		\end{axiom}
		Das Auswahlaxiom ist intuitiv "offensichtlich", ist jedoch aus den restlichen Axiomen der sogenannten Zermelo-Fraenkelschen Mengentheorie (ZF) nicht beweisbar. Man kann zeigen
		\begin{theorem}
			Ist ZF widerspruchsfrei, so auch sowohl ZF + AC als auch ZF + $\lnot$AC
		\end{theorem}
		Das heißt, es existieren Modelle von ZF, in denen AC beweisbar ist und es gibt andere Modelle von ZF in denen $\lnot$AC beweisbar ist. Damit können sie nun ZF akzeptieren oder ablehnen. Wenn sie es ablehnen, kommen sie aber nicht sehr weit. Deshalb akzeptieren wir es.
		AC sagt, dass man aus "beliebig vielen" Mengen jeweils ein Element wählen kann, egal wie groß die Indexmenge auch sein mag. Dies ist nicht der Ort, um zu sehr zu problematisieren/vertiefen. Der Pragmatiker sieht doch, dass AC wahr sein muss. Oder? Warten Sie...
		\begin{definition}
			\begin{enumerate}
				\item 
					Eine binäre Relation $\leq$ auf $X$ heißt Halbordnung, falls
					\begin{itemize}
						\item 
							$\forall_{x \in X} x \leq x$ (reflexiv)
						\item
							$\forall_{x,yz \in X} x \leq y \land  y \leq z \Rightarrow x \leq z$(transitiv)
						\item
							$\forall_{x,y \in X} x \leq y \land y \leq x \Rightarrow x =y$
					\end{itemize}
					Beachte, dass es $x,y \in X$ geben kann, für die weder $x \leq y$ noch $y \leq x$.
				\item
					$(X, \leq)$ heißt Kette oder totalgeordnete Menge, falls $\forall_{x,y \in X} x \leq y \lor y \leq x$.
			\end{enumerate}
		\end{definition}
		\begin{example}
			\begin{enumerate}
				\item
					$X$ Menge, $(\mathcal{P}(X), \subset)$ ist halbgeordnet.
				\item
					$\mathbb{Z}, \mathbb{R}$ mit den üblichen $\leq$ sind total geordnet. 
				\item 
					$(X, \leq)$ heißt wohlgeordnet, falls $(X, \leq)$ total geordnet und jede Teilmenge $A \subset X$ ein kleinstes Element besitzt.
					\begin{itemize}
						\item
							$(\mathbb{N}, \leq)$ ist wohlgeordnet.
						\item
							$(\mathbb{R}_+, \leq)$ ist nicht wohlgeordnet
					\end{itemize}
			\end{enumerate}
		\end{example}
		\begin{theorem}{(Wohlordnungssatz von Zermelo)(WO)}
			Jede Menge $X \neq \emptyset$ kann wohlgeordnet werden.
		\end{theorem}
		\begin{proof}
			Eine \textbf{sehr hässliche} Induktion.
		\end{proof}
		WO ist stärker als AC:
		\begin{theorem}{(In ZF)}
			WO $\Rightarrow$ AC
		\end{theorem}
		\begin{remark}
			Unter ZF wollen wir hier die bisher verwendete naive Mengenlehre (ohne sie exakt in Axiomen zu formulieren) verstehen. Sie werden im Beweis genau sehen, welche Mengenoperationen verwendet werden. Der Satz sagt also, dass innerhalb dieses Rahmens der naiven Mengenlehre der WO das AC impliziert.
		\end{remark}
		\begin{proof}
			Seien also $(X_i)_{i \in I}$ Mengen, $X= \bigcup_{i \in I} X_i, \ \forall_{i} X_i \neq \emptyset$. Sei (WO) $\leq$ eine Wohlordnung auf $X$. Für $U \in \mathcal{X}, U \neq \emptyset$ setze
			\begin{equation}
				\mathcal{M}(U) := \min \{ x \in X | x \in U \}
			\end{equation}
			$\mathcal{M}(U)$ existiert, da $\leq$ eine Wohlordnung ist. Offenbar ist dann
			\begin{equation}
				\begin{split}
					f: I &\rightarrow X \\
					f(i) &:= \mathcal{M}(X_i)
				\end{split}
			\end{equation}
			eine Auswahlfunktion.
		\end{proof}
		Als nächstes wenden wir uns Maximalitätsprinzipien zu. Diese gehen zurück auf Hausdorff, Zermelo, Zorn
		\begin{definition}
			Sei $(X, \leq)$ halbgeordnet. $\bar{x} \in X$ heißt maximal, falls $\forall_{y \in X} y \leq x \Rightarrow y = x$. Mit anderen Worten gibt es bezüglich $\leq$ keine echt größeren Elemente.
		\end{definition}
		\begin{attention}
			Es kann mehrere maximale Elemente geben. Sei
			\begin{equation}
				X = \left\{ \{ 1\}, \{ 1,2 \}, \{ 3,4 \}, \{ 5 \} \right\}
			\end{equation}
			mit $\subset$ als Halbordnung. \\
			Dann sind $\{ 1,2 \}, \{ 3,4 \}, \{ 5 \}$ maximal.
		\end{attention}
		Ist $(X, \leq)$ halbgeordnet und sei $\kappa_X = \{ K \subset X | K \ Kette \} \subset \mathcal{P}(X)$. Dann ist $(\kappa_X, \subset)$ ebenfalls halbgeordnet. Wir betrachten die Ketten stets mit dieser Ordnung.
		\begin{lemma}
			Sei $\mathcal{C} \subset \kappa_X$ eine Kette (also eine Kette von Ketten in $X$). Dann ist $\bigcup \mathcal{C}$ eine Kette in $X$.
		\end{lemma}
		\begin{theorem}{(Hausdorffscher Maximalkettensatz)(HMK)}
			Sei $(X, \leq), X \neq \emptyset$. Dann existiert eine maximale Kette $K \subset X$. Das heßt $(\kappa_X, \subset)$ besitzt maximale Elemente.
		\end{theorem}
		Für endliche Mengen ist das klar. Man starte mit irgendeinem Element $x \in X$ und der Kette $\{ x \}$. Falls $\{ x \}$ nicht maximal, so gibt es eine Kette $K_i \subset \{ x \}$. So fahre man induktiv fort... Aber was ist mit unendlichen Mengen?? 
		\begin{lemma}{(Spezielles Lemma von Zorn)(ZLS)}
			Sei $\emptyset \neq \xi \subset \mathcal{P}(X)$ ein nichtleeres Mengensytem (d.h. $\xi \neq 0$ ist Teilmenge der Potenzmenge einer Menge $X$). $\Xi$ sei durch $\subset$ teilweise geordnet. Für jede Kette $K \subset \xi$ sei $\bigcup K \in \xi$. Dann besitzt $\xi$ maximale Elemente.
		\end{lemma}
		\begin{theorem}
			HMK $\Rightarrow$ ZLS
		\end{theorem}
		\begin{proof}
			Sei $\xi$ wie in ZLS gegeben. \\
			Aus HMK folgt, dass maximale Ketten $K_{max} \subset \xi$ gibt. Sei $\bar{x}= \bigcup K_{max}$. Nach der Vorraussetzung ist $\bar{x} \in \xi$. Wäre $\bar{x}$ nicht maximal, so gäbe es $y \in \xi$ mit $y \supsetneq \bar{x}$. Dann wäre aber $K_{max} \cup \{ y\}$ eine Kette, welche echt größer als $K_{max}$ wäre. Widerspruch!
			\begin{attention}
				Falls $z \in K_{max}$, so ist $z \subset \bar{x} \subset y$, Also ist $y \notin K_{max}$ und größer als jedes Element von $K_{max}$
			\end{attention}
		\end{proof}
		\begin{lemma}{(Lemma von Zorn)(ZL)}
			Sei $(X, \leq), X \neq 0$ halbgeordnet. Jede Kette $K \subset X$ besitze eine obere Schranke. Dann existiert zu jedem $\alpha \in X$ ein maximales Element $x_{\alpha} \geq \alpha$ in $X$.
		\end{lemma}
		\begin{theorem}{(in ZF)}
			ZLS $\Rightarrow$ ZL
		\end{theorem}
		\begin{proof}
			Sei $\kappa_{\alpha} = \{ K \subset X | K \ Kette \, \alpha \in K \}$. $\{ \alpha \}$ ist sicherlich eine Kette in $X$. Also ist $\kappa_{\alpha} \neq \emptyset$ und durch $\subset$ halbgeordnet. \\
			Sei $\mathcal{C} \subset \kappa_{\alpha}$ eine Kette. Nach Lemma 8.7.4 ist dann auch $\bigcup \mathcal{C}$ eine Kette in $X$. Auf $(\kappa_{\alpha}), \subset$ ist also ZLS anwendbar und es folgt die Existenz eines maximalen Elements $K \in \kappa_{\alpha}$. Das heißt $K$ ist eine maximale Kette in $X$ und $\alpha \in K$. Nach Voraussetzung existiert eine obere Schranke $\bar{x} \in X$ von $K$. D.h. $\bar{x} \in X$ und $\forall_{k \in K} k \leq \bar{x}$. Wäre $\bar{x}$ nicht maximal, so gäbe es $y \in X$ mit $y > \bar{y}$. Folglich $y \notin K$. Also wäre $K \cup \{ y \}$ eine echt größere Kette als $K$. Widerspruch.
		\end{proof}
		\begin{theorem}{(In ZF)}
			ZL $\Rightarrow$ HMK
		\end{theorem}
		\begin{proof}
			Sei $(X, \leq)$ halbgeordnet und $\neq \emptyset$. $(\kappa_X, \subset)$ ist bekanntlich ebenfalls $\neq \emptyset$ und halbgeordnet. Ist $K \subset \kappa_X$ eine Kette, so ist nach Lemma (8.7.4) die Vereinigung $\bigcup K \in \kappa_X$ ebenfalls eine Kette. $\bigcup K$ ist sicherlich eine obere Schranke für K. Nach ZL existiert also ein maximales Element in $\kappa_X$
		\end{proof}
		\textbf{Zwischenergebnis (in ZF)} \\
		\begin{center}
			\begin{tikzcd}
				& HMK \arrow[dr, Leftrightarrow] \arrow[dl, Leftrightarrow] & & & WO  \arrow[d, Rightarrow]\\
				ZLS \arrow[rr, Leftrightarrow] & & ZL  \arrow[rr, dash, "?"] \arrow[rru, dash, "?"]& & AC
			\end{tikzcd}
		\end{center}
		\begin{theorem}{(in ZF)}
			ZL $\Rightarrow$ WO
		\end{theorem}
		\begin{proof}
			Sei $X \neq \emptyset$ eine Menge. Betrachte Paare $(U, \leq), U \subset X$, $\leq$ Wohlordnung auf $U$. \\
			Bezeichne mit $\xi$ die Menge aller solchen Paare.
			\begin{itemize}
				\item 
					$\xi \neq \emptyset$: Sei $x \in X$. Dann ist $U = \{ x \} \neq \emptyset$ und $U \times U$ eine Wohlordnung auf $U$.
				\item 
					Halbordnung auf $\xi$: $(U, \leq_U) \leq_{\xi} (V, \leq_V)$, falls
					\begin{enumerate}
						\item 
							$U \subset V$
						\item 
							$\leq_V |_U = \leq_U$
					\end{enumerate}
				\item 
					$(\xi, \leq_{\xi})$ erfüllt Voraussetzung von ZL: Sei $K \subset \xi$ eine Kette, $\hat{U}:= \bigcup K$. Sicherlich gilt $\forall_{U \in K} U \subset \hat{U}$. Für $x,y \in \hat{U}$ wähle man $U_x, U_y \in K$ mit $x \in U_x, y \in U_y$. Da $K$ eine Kette ist, sei ohne Beschränkung der Allgemeinheit $U_x \subset U_y$, also $x, y \in U_y$. Setze $x \leq y :\Leftrightarrow x \leq_{U_y} y$. \\
					Dies erklärt unabhängig von der Wahl von $U_x, U_y$ eine Wohlordnung (Übung) auf $\hat{U}$. Offenbar ist $(\hat{U}, \leq)$ eine obere Schranke für $K$.
			\end{itemize}
			Wir wenden ZL an und erhalten ein maximales Element $(\bar{U}, \leq)$ in $\xi$.
			\begin{claim*}
				$\bar{U}=X$, folglich ist $\leq_{\bar{U}}$ eine Wohlordnung auf $X$.
			\end{claim*}
			\begin{proof}
				Wäre $\bar{U} \neq X$, so würde ein $\bar{x} \in X - \bar{U}$ existieren. Dann ist $V := \bar{U} \cup \{ \bar{x} \}$ mit
				\begin{equation}
					z \leq_V w :\Leftrightarrow z \leq_{\bar{U}} w, \ z,w \in \bar{U}
				\end{equation}
				und $\bar{x} \leq z$ für alle $z \in \bar{U}$ eine Wohlordnung auf $V$. Dies widerspricht der Maximalität von $(\bar{U}, \leq)$
			\end{proof}
			Dies beendet den Beweis.
		\end{proof}
\end{document}